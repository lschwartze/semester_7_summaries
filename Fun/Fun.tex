\documentclass[a4paper, 12pt]{article}

\usepackage{fullpage}
\usepackage[utf8]{inputenc}
\usepackage[english]{babel}
\usepackage{amsmath,amssymb}
\usepackage[explicit]{titlesec}
\usepackage{ulem}
\usepackage[onehalfspacing]{setspace}
\usepackage{amsthm}

\theoremstyle{plain}
\newtheorem{theorem}{Theorem}[subsection] % reset theorem numbering for each chapter

\theoremstyle{definition}
\newtheorem{definition}[theorem]{Definition} % definition numbers are dependent on theorem numbers
\theoremstyle{lemma}
\newtheorem{lemma}[theorem]{Lemma}

\theoremstyle{remark}
\newtheorem{remark}[theorem]{Remark}

\theoremstyle{corollary}
\newtheorem{corollary}[theorem]{Corollary}

\theoremstyle{example}
\newtheorem{example}[theorem]{Example}

\titleformat{\subsection}
{\small}{\thesubsection}{1em}{\uline{#1}}
\begin{document}
	\begin{titlepage} 
		\title{Fun Summary}
		\clearpage\maketitle
		\thispagestyle{empty}
	\end{titlepage}
	\tableofcontents
	\newpage
	\section{metric spaces}
	\label{sec: metric spaces}
	\subsection{metric spaces}
	\begin{definition}
		A \underline{metric space} is a non-empty set $X$ together with a map \[d: X \times X \to \mathbb{R}\]
		\[(x,y) \mapsto d(x,y)\]
		such that \begin{enumerate}
			\item $d(x,y) = 0$ iff $x = y$
			\item $d(x,y) = d(y,x)$
			\item $d(x,z) \leq d(x,y) + d(y,z) \; \forall x,y,z \in X$
		\end{enumerate}
	\end{definition}

	\begin{remark}($d$ admits only positive values)\\
		\[0 = d(x,x) \leq d(x,y) + d(y,x) = 2d(x,y)\]
	\end{remark}

	\begin{example}
		\begin{enumerate}
			\item $d_2(x,y) = ||x-y||_2$
			\item $d(x,y) = \begin{cases}
				0 \; \text{ if } x = y\\
				1 \; \text{ else}
			\end{cases}$
		\end{enumerate}
	\end{example}
	
	\begin{definition} (convergence)\\
		A sequence $(x_n)_{n \in \mathbb{N}}$ in a metric space $(X,d)$ is said to be convergent to $x \in X$ if \[x_n \to x \text{ in } (X,d)\] or \[\lim\limits_{n \to \infty} x_n = x \text{ in } (x,d)\]
	\end{definition}

	\subsection{Topology in metric spaces}
	Let $(X,d)$ be a metric space.
	\begin{definition}
		\begin{enumerate}
			\item an open ball is defined by \[B_r(x) = \{y \in X: \; d(x,y) < r\}\]
			\item $O \subset X$ is called open if $\forall y \in O$ there is $r > 0$ such that $B_r(y) \subset O$
			\item $A \subset X$ is closed if $X \setminus A$ is open.
		\end{enumerate}
	\end{definition}

	\begin{theorem} (metric spaces are topological spaces)\\
		Let $\mathcal{T}$ be the set of open subsets of $X$. Then \begin{enumerate}
			\item $\varnothing, X \in \mathcal{T}$
			\item if $U,V \in \mathcal{T}$, then $U \cup V \in \mathcal{T}$
			\item if $\{U_i\}_{i \in I} \subset \mathcal{T}$, then $\bigcup_{i \in I} \in \mathcal{T}$
		\end{enumerate}
	\end{theorem}

	\begin{remark}
		\begin{enumerate}
			\item $\varnothing, X$ are closed
			\item finite union of closed sets is closed
			\item arbitrary intersections of closed sets is closed
		\end{enumerate}
	\end{remark}

	\begin{lemma}
		$A \subset X$ is closed iff $\forall$ convergent sequences $(x_n)_{n \in \mathbb{N}} \subset A$ the limit point is in $A$.
	\end{lemma}

	\begin{definition}
		For $M \subset X$ we define \[\overline{M} = \bigcap\limits_{A \supset M, \; A \text{ closed}}\] as the closure of $M$ and \[\overset{°}{M} = \bigcup\limits_{O \subset M, \; O \text{ open}}\] as the interior of $M$.\\
		$\partial M = \overline{M} \setminus \overset{°}{M}$ is the boundary of $M$
	\end{definition}

	\noindent\underline{Attention:}\\
	Define the closed ball as $\overline{B}_r(a) = \{y \in X: \; d(y,a) \leq r\}$. Then in general $\overline{B_r(a)} \neq \overline{B}_r(a)$.\\
	Example: Take $X \neq \varnothing$ and the trivial metric $d$. Then \[B_1(a) = \{a\} = \overline{B_1(a)}\] but $\overline{B}_1(a) = X$.
	
	\subsection{separability and completion}
	Let $(X,d)$ be a metric space.
	\begin{definition}
		\begin{enumerate}
			\item $M \subset X$ is called dense in $X$ if $\overline{M} = X$.
			\item $X$ is called separable if $X$ has a countable dense subset.
		\end{enumerate}
	\end{definition}

	\begin{remark}
		$M$ is dens in $X$ iff \[\forall x \in X \; \forall \varepsilon > 0 \; \exists y \in M \text{ s.t. } d(x,y) < \varepsilon\]
	\end{remark}

	\begin{definition}
		\begin{enumerate}
			\item $(x_n)_{n \in \mathbb{N}} \subset X$ is called a Cauchy sequence if \[\forall \varepsilon > 0 \; \exists N \in \mathbb{N} \text{ s.t. } m,n > N \text{ implies } d(x_n,x_m) < \varepsilon\]
			\item A metric space in which all Cauchy sequences converge is called complete.
		\end{enumerate}
	\end{definition}

	\begin{example}
		\begin{enumerate}
			\item $(C^0([a,b], \mathbb{R}), d_\infty)$ with $d_\infty(f,g) = \max\limits_{x \in [a,b]} \left|f(x)-g(x)\right|$ is complete.
			\item $(\mathbb{R}^n, d_2)$ with $d_2(x,y) = ||x-y||_2$ is complete.
		\end{enumerate}
	\end{example}

	\begin{lemma}
		Let $(X,d)$ be a complete metric space and $\varnothing \neq A \subset X$. Then $(A,d)$ is complete iff $A$ is closed.
	\end{lemma}

	\begin{definition}
		$A \subset X$ is called bounded if its diameter \[diam(A) = \sup\{d(x,y): \; x,y \in A\}\] is finite.
	\end{definition}

	\begin{theorem}
		$(X,d)$ is complete iff $\forall (F_n)_{n \in \mathbb{N}}$ sequences of closed subsets such that $F_{n+1} \subset F_n$ and $diam(F_n) \to 0$ then \[\exists ! x_0 \in X \text{ s.t.} \bigcap\limits_{n \in \mathbb{N} F_n = \{x_0\}}\]
	\end{theorem}

	\subsection{Continuity}
	\begin{definition}
		Let $(X,d_x), (Y, d_y)$ be metric spaces and $f: X \to Y$. $f$ is continuous in $x_0$ if \[\forall \varepsilon > 0 \; \exists \delta > 0 \text{ s.t. } \forall x \in X \; d_x(x,x_0) < \delta \text{ implies } d_y(f(x),f(x_0)) < \varepsilon\]
		With sequences: \[\forall (x_n)_{n\in \mathbb{N}} \subset X \; x_n \to x_0 \text{ in } (X,d_x) \text{ if it holds } (f(x_n))_{n \in \mathbb{N}} \subset Y, \, f(x_n) \to f(x_0) \text{ in } (Y,d_y)\]
	\end{definition}
	$f$ is continuous if $f$ is continuous in $x_0$ for all $x_0 \in X$.\\
	In other words $f$ is continuous if for all $O \subset Y$ open (closed) $f^{-1}(O)$ is open (closed) in $X$.\\
	\underline{Special case}: $f$ is Lipschitz continuous if $\exists L > 0$ s.t. \[d_y(f(x),f(y)) \leq Ld_x(x,y) \; \forall x,y \in X\]
	$f$ is an isometric if $\forall x,y \in X$ it holds that $d_Y(f(y),f(x)) = d_x(x,y)$.
	\subsection{Compact sets}
	\begin{definition}
		Let $(X,d)$ be a metric space and $A \subset X$. 
		\begin{enumerate}
			\item an open cover of $A$ is a collection $\{U_i\}_{i \in I}$ where $I\neq \varnothing$ is an arbitrary index set of open subsets of $X$ s.t. $A \subset \bigcup\limits_{i \in I} U_i$.
			\item $A$ is compact if every open cover of $A$ contains a finite subcover i.e. there is $N \in \mathbb{N}$ and indices $i_1,...,i_N$ such that \[A \subset U_1 \cup ... \cup U_N\]
			\item $A$ is sequentially compact if every sequence in $A$ has a convergence subsequence in $A$.
			\item $A$ is called precompact or totally bounded if $\forall \varepsilon > 0 \; \exists N \in \mathbb{N}$ and $\exists x_1,...,x_N \in X$ such that $A \subset \bigcup_{i = 1}^N B_\varepsilon(x_i)$.
		\end{enumerate}
	\end{definition}

	\begin{theorem}
		Let $(X,d)$ be a metric scape and $A \subset X$. The following are equivalent:
		\begin{enumerate}
			\item $A$ is compact
			\item $A$ is sequentially compact
			\item $(A,d)$ is complete and $A$ is precompact.
		\end{enumerate}
	\end{theorem}

	\begin{remark}
		If $A$ is precompact, then $\overline{A}$ is precompact. Further, if $(X,d)$ is complete and $A\subset X$ then $A$ is precompact $\Leftrightarrow$ $\overline{A}$ is compact.
	\end{remark}

	Recall: $A$ compact $\Rightarrow$ bounded and closed and $f: X \to Y$ continuous with $A \subset X$ compact, then $f(A)$ is compact as well. Further, if $f:A \to \mathbb{R}$ is continuous and $A$ is compact, then \[\exists x_1, x_2 \in A \; \text{ s.t. } f(x_1) \leq f(x) \leq f(x_2) \; \forall x \in A\]
	Theorem of Heine-Borel: $A \subset \mathbb{R}^n$ is compact iff $A$ is closed and bounded.
	\subsection{Theorem of Baire}
	\begin{theorem}
		\label{th: baire}
		Let $(X,d)$ be a complete metric space and $\forall n \in \mathbb{N}$ consider $U_n \subset X$ open and dense. Then \[\bigcap\limits_{n \in \mathbb{N}} U_n\] is dense in $X$.
	\end{theorem}

	\begin{remark}
		\begin{enumerate}
			\item Completeness is in general necessary. Consider $(\mathbb{Q}, d)$ and $d(x,y) = \left|x-y\right|$. Define a sequence $x_n$ such that $\mathbb{Q} = \{x_n \; n \in \mathbb{N}\}$. Take $U_n = \mathbb{Q} \setminus \{x_n\}$ which is open and dense. Then \[\bigcap\limits_{n \in \mathbb{N}} U_n = \varnothing\]
		\end{enumerate}
	\end{remark}

	\begin{corollary}
		Let $(X,d)$ be a complete metric space. Let $\forall n \in \mathbb{N}$, $A_n \subset X$ be closed and \[X = \bigcup\limits_{n \in \mathbb{N}} A_n\] Then $\exists N \in \mathbb{N}$ s.t. $A_N$ has an interior point.
	\end{corollary}

	\begin{remark}
		Theorem \ref{th: baire} is also called \underline{Baire category theory}.\\
		\begin{itemize}
			\item In a metric space $(X,d)$ $A \subset X$ is called nowhere dense if $\overline{A}$ has no interior points.
			\item $A$ is called of first category if $\exists (M_n)_{n \in \mathbb{N}}$ where $M_n \subset A$ nowhere dense s.t. $A = \bigcup_{n \in \mathbb{N}} M_n$
			\item $A$ is called of second category if it is not of first category
		\end{itemize}
		Hence the theorem of Baire implies that every complete metric space is of second category.
	\end{remark}
	
	\section{Normal spaces and Banach spaces}
	Let $X$ be a $\mathbb{K}$-vector space where $\mathbb{K} = \mathbb{R}$ or $\mathbb{C}$.
	\subsection{definitions}
	\begin{definition}
		A map $||\cdot||: X \to \mathbb{R}$ is called a norm on $X$ if \begin{enumerate}
			\item $\forall x \in X$, $||x||\geq 0$ and $||x|| = 0$ iff $x = 0$
			\item $\forall \lambda \in \mathbb{K}$ and $\forall x \in X$ it holds that $||\lambda x|| = \left|\lambda\right|\cdot ||x||$
			\item $\forall x, y \in X$ it holds $||x+y|| \leq ||x|| + ||y||$
		\end{enumerate}
		The pair $(X,||\cdot||)$ is called an normed space.\\
		
		\noindent $p:X \to \mathbb{R}$ is called a seminorm if $p(x) \geq 0$ $\forall x \in X$ and 2. and 3. are also satisfied.
	\end{definition}

	\begin{example}
		\begin{enumerate}
			\item $C^0([0,1]; \mathbb{R})$ with $||f||_\infty = \max\limits_{x \in [0,1]} \left|f(x)\right|$
			\item more general for a compact metric space $K$: $C^0(K,\mathbb{R})$ with $||f||_\infty = \max\limits_{x \in K} \left|f(x)\right|$
			\item $C^1([0,1]; \mathbb{R})$ with $p(f) = \max\limits_{x \in [0,1]} \left|f'(x)\right|$
			\item $\Omega \subset \mathbb{R}^n$ measurable. $L^1(\Omega) = \{f: \Omega \to \mathbb{R}: \; f \text{ integrable }\}$ with \[p: L^(\Omega) \to \mathbb{R}: \; p(f) = \int_{\Omega} \left|f(x)\right|dx\] then $p$ is a seminorm.
		\end{enumerate}
	\end{example}
	\begin{remark}
		Any normed space is a metric space via \[d(x,y) = ||x-y||\] All concepts from chapter \ref{sec: metric spaces} apply.
	\end{remark}

	\begin{lemma}
		Let $(X,||\cdot||)$ be a normed space. Then $X$ is called separable iff $\exists A \subset X$ countable such that s.t. $\overline{span\{A\}} = X$ where $span\{A\} = \{\sum_{i=1}^{n} \lambda_i x_i$\} with $n \in \mathbb{N}$, $\lambda_i \in K$ and $x_i \in A$. Here the colusre is defined w.r.t the norm.
	\end{lemma}

	\begin{definition}
		A complete normed space is called a Banach space.
	\end{definition}

	\subsection{Example: $l^p$-spaces}
	We consider the vector space $\mathbb{K}^\mathbb{N}$ of sequences in in $\mathbb{K}$. Let $x = (x_n)_{n \in \mathbb{N}}$ and $y = (y_n)_{n \in \mathbb{N}}$. Define $x+y = (x_n+y_n)_{n \in \mathbb{N}}$ and $\lambda x = (\lambda x_n)_{n \in \mathbb{N}}$.\\
	For $x \in \mathbb{K}^\mathbb{N}$ define \[||x||_{l^p} = \left(\sum_{n=1}^\infty \left|x\right|^p\right)^{1/p}\] for $1\leq p < \infty$ and \[||x||_{l^\infty} = \sup\limits_{n \in \mathbb{N}} \left|x_n\right|\] else.\\
	Define $l^p = \{x = (x_n)_{n \in \mathbb{N}}: ||x||_{l^p} < \infty\}$ for $1 \leq p \leq \infty$. We find that $l^p$ is a subspace of $\mathbb{K}^\mathbb{N}$ and $l^p$ is a normed space (for the triangle inequality use the Hölder inequality).
	\begin{theorem}
		For $1 \leq p \leq \infty$ $l^p$ is a Banach space.
	\end{theorem}
	\begin{lemma}
		For finite $p$, $l^p$ is separable while $l^\infty$ is not.
	\end{lemma}

	\subsection{Finite dimensional normed spaces}
	Let $X$ be a vector space over $\mathbb{K}$. $\exists e_1,...,e_n \in X$ s.t. \[\forall x \in X; \; \exists \lambda_1,...,\lambda_n \in \mathbb{K}: \; x = \sum_{i=1}^n \lambda_i x_i\]
	For $p \in [1,\infty)$ we define \[||x||_p = \left(\sum_{i=1}^n \left|\lambda_i\right|^p\right)^{1/p}\] and for $p = \infty$ \[||x||_\infty = \max\limits_{1 \leq i \leq n} \left|\lambda_i\right|\]
	
	\begin{definition}
		Two norms are equivalent in that \[\alpha ||\cdot ||_1 \leq ||\cdot ||_2 \leq \beta ||\cdot ||_1\]
	\end{definition}

	\begin{theorem}
		In a finite dimensional space, all norms are equivalent.
	\end{theorem}

	\begin{theorem}
		Finite dimensional normed spaces are Banach spaces.
	\end{theorem}

	\subsection{On the closure of $\overline{B_1(0)}$}
	\begin{lemma}[Lemma of Riesz, Lemma of the almost orthogonal element]
		Let $X$ be a normed space. $U \subset X$ a closed subspace of $X$ s.t. $U \neq X$. Then $\forall \lambda \in (0,1) \exists x_\lambda \in X$ s.t. $||x_\lambda|| = 1$ and $dist(x_\lambda, U) \geq \lambda$.
	\end{lemma}
	\begin{theorem}
		In a normed space $X$, $\overline{B_1(0)}$ is compact iff $X$ is finite dimensional.
	\end{theorem}

	\section{A question from approximation theory}
	\subsection{Theorem of Stone-Weierstrass}
	Let $X$ be a compact metric space. Then $(C^0(X), \mathbb{K}), ||\cdot ||_\infty$, where $||f||_\infty = \max\limits_{x \in X} \left|f(x)\right|$ is a Banach space.\\
	Which property of $A \subset C^0(X, \mathbb{K})$ ensures that $A$ is dense.
	\begin{definition}
		$A \subset C^0(X, \mathbb{K})$ is called subalgebra, if $\forall f,g, \in A$ \begin{enumerate}
			\item $\lambda f + \mu g \in A$ (subspace)
			\item $f\cdot g \in A$
		\end{enumerate}
	\end{definition}  
	\begin{example}
		\begin{itemize}
			\item $\{p: [0,1] \to \mathbb{R}\}$ is a subalgebra of $C^0([0,1]; \mathbb{R})$. 
			\item $\{f: [-1,1] \to \mathbb{R}; f\text{ continuous and even}\}$ is a subalgebra.
		\end{itemize}
	\end{example}
	\begin{remark}
		If $A$ is a subalgebra, then $\overline{A}$ is also a subalgebra.
	\end{remark}
	\begin{definition}
		Let $A \subset C^0(X)$ be a subalgebra. \begin{enumerate}
			\item $A$ is called unital if $1 \in A$
			\item $A$ separates point if $x,y \in X$, $x \neq y$, $\exists f \in A$ s.t. $f(x) \neq f(y)$.
			\item (if $\mathbb{K} = \mathbb{C}$) $A$ is stable under conjuguation if from $f \in A$ we conclude that also $\overline{f} \in A$.
		\end{enumerate}
	\end{definition}
	\begin{remark}
		If $A$ is unital then all constant functions are in $A$.
	\end{remark}
	\begin{lemma}
		Consider $f: [-1,1] \to \mathbb{R}$ where $f(x) = \left|x\right|$. Then $\exists$ sequence of polynomials $(p_n)_{n \in \mathbb{N}}$ s.t. \[p_n \to f\] uniformly in $[-1,1]$.
	\end{lemma}
	\begin{lemma}
		Let $A \subset C^0(X, \mathbb{R})$ be a unital subalgebra. Then \begin{enumerate}
			\item if $f \in A$ then $\left|f\right| \in \overline{A}$.
			\item if $f,g \in A$ then $\max\{f,g\} \in \overline{A}$ and $\min\{f,g\} \in \overline{A}$
		\end{enumerate}
	\end{lemma}

	\begin{theorem}[Stone-Weierstrass]
		Let $A$ be a compact metric space. $A \subset C^0(X,\mathbb{K})$ is a unital subalgebra that separates points and if $\mathbb{K} = \mathbb{C}$ is stable under conjugation, then $A$ is dense in $C^0(X,\mathbb{K})$ w.r.t $||\cdot||_\infty$.
	\end{theorem}
	\subsection{Applications}
	\begin{theorem}[Theorem of Weierstraß]
		Let $[a,b]$ be a compact interval in $\mathbb{R}$, $f:[a,b] \to \mathbb{R}$ be a continuous function and $\varepsilon>0$. Then $\exists p: [a,b] \to \mathbb{R}$ a polynomial s.t. \[||p-f||_\infty = \sup_{x \in [a,b]} \left|p(x)-f(x)\right| < \varepsilon\]
	\end{theorem}
	\begin{definition}
		A function $f:\mathbb{R} \to \mathbb{C}$ is periodic if \[f(x+t) = f(x)\] for a $t \in \mathbb{R}$ and all $x \in \mathbb{R}$.
	\end{definition}
	\begin{remark}
		If $f$ is periodic with period $t$ then $\tilde{f}: \mathbb{R} \to \mathbb{C}$ where $\tilde{f}(x) = f(t\frac{x}{2\pi})$ is periodic of period $2\pi$.\\
		Consider $C_{2\pi}^0(\mathbb{R}, \mathbb{C})$ the space of continuous $2\pi$-periodic functions. We consider the span of $\{e^{ikx} = \cos(kx)+i\sin(kx),k \in \mathbb{Z}\}$.
	\end{remark}
	\begin{definition}
		A trigonometric polynomial is a function $f:\mathbb{R} \to \mathbb{C}$ \[f(x) = \sum_{k=-N}^{N} c_k\cdot e^{ikx}\] with $c_k \in \mathbb{C}$
	\end{definition}
	\begin{theorem}[Approximation of periodic functions]
		Trigonometric polynomials are dense in $(C_{2\pi}^0(\mathbb{R},\mathbb{C}),||\cdot||_\infty)$ 
	\end{theorem}
	\underline{Application to neural networks}\\
	The simplest case of a neural network has $d$ inputs $x_1,...,x_d$ and one output $Z$ called a \textit{feed forward} network. Each input influences the output and $x_i$ might have a weight $\alpha_i$ associated to it. The output is a function in $x=(x_1,...,x_d)$ and the weights $\alpha= (\alpha_1,...,\alpha_d)$. For instance, the output is often of the form \[Z = \sum_{i=1}^d \alpha_ix_i + b\] where $b$ is the bias of the network. To make the network slightly stronger, we add a intermediate layer $y = (y_1,...,y_r)$ where each $x_i$ is connected to each $y_j$ with the associated weight $\gamma_{i,j}$. The $y$ layer (often called activation) is the connected to the output $Z$ as above with weights $\alpha_j$. We introduce the realtion \[y_j = \Phi(\sum_{i=1}^d \gamma_{j,i}x_i+b)\] for a measurable function $\Phi$. Lastly, the output is then given by \[Z = \sum_{j=1}^r \alpha_j y_j\]
	\begin{definition}
		\begin{enumerate}
			\item $A^d = \{a: \mathbb{R}^d \to \mathbb{R}: \; a(x9 = w^Tx+b)\}$ where $w \in \mathbb{R}^d$ and $b \in \mathbb{R}$.
			\item given $\Phi: \mathbb{R} \to \mathbb{R}$ measurable $d \in \mathbb{N}$ define $\Sigma^d(\Phi) = \{f: \mathbb{R}^d \to \mathbb{R}: \; f(x) = \sum_{j=1}^N \alpha_j \Phi(a_j(x)) \text{ wtih } N \in \mathbb{N}, \alpha_j \in \mathbb{R}, a_j \in A^d\}$ as the set of single hidden layer feed forward networks.
			\item A squashing function is a measurable non-decreasing function $\Phi: \mathbb{R} \to \mathbb{R}$ s.t. $\lim_{x \to -\infty} \Phi(x) = 0$ and $\lim_{x \to \infty} \Phi(x) = 1$.
		\end{enumerate}
	\end{definition}
	\begin{theorem}[Universal Approximation theorem of Hornik-Stinchcombe-White]
		Let $\Phi$ we a squashing function $K\subset \mathbb{R}^d$ compact $f:K \to \mathbb{R}$ continuous and $\varepsilon > 0$. Then $\exists g \in \Sigma^d(\Phi)$ s.t. \[\sup_{x \in K} \left|f(x)-g(x)\right| < \varepsilon\]
	\end{theorem}

\section{Continuous linear maps}
	$(X,||\cdot ||_X), (Y,||\cdot ||:Y)$ are $K$-Vector spaces with $K = \mathbb{R}$ or $K = \mathbb{C}$. $T:X \to Y$ is called linear if \[T(\lambda_1x_1 + \lambda_2 x_2) = \lambda_1 T(x_1) + \lambda_2 T(x_2)\]
	\subsection{Continuity of linear maps}
	\begin{definition}
		LEt $T:X\to Y$ be linear. Then $T$ is bounded if $\exists C>0$ s.t. \[||Tx||_Y \leq C||x||_X \; \forall x \in X\]
		or equivalently \[\sup_{x \in X\setminus \{0\}} \frac{||Tx||_Y}{||x||_X} \leq C\] which is also equivalent to \[\sup_{x \in X, ||x||_X=1} ||Tx||_Y \leq C\]
	\end{definition}
	\begin{theorem}
		For $T: X \to Y$ linear, the following are equivalent: \begin{enumerate}
			\item $T$ is continuous 
			\item $T$ is continuous in 0
			\item $t$ is bounded
		\end{enumerate}
	\end{theorem}
	\begin{lemma}
		Let $X$ have infinite dimension. Then $\exists T: X \to \mathbb{K}$ linear and not bounded.
	\end{lemma}
	\begin{definition}
		Define $L(X,Y)$ as the set of linear continuous ($\Leftrightarrow$ bounded) maps from $X$ to $Y$. With the usual addition $((T_1+T_2)(x) = T_1(X) + T_2(x))$ and the scalar multiplication $((\lambda(T)(x)) = \lambda T(x))$ this is a vector space.\\
		If $X=Y$ we write $L(X)$. For $T \in L(X,Y)$ \[\ker T = \{x \in X: Tx = 0\}\] and \[\Im(T) = \{y \in Y: \exists x\in X: \; Tx = y\}\]
	\end{definition}
	\subsection{Operatornorm and dual space}
	\begin{theorem}
		Let $X \neq \{0\}$. \begin{itemize}
			\item $L(X,Y)$ with the operatornorm $||T|| = \sup_{x \in X\setminus \{0\}} \frac{||Tx||_Y}{||x||_X} = \sup_{x \in X, ||x||_X=1} ||Tx||_Y$ is a normed space. We have \[||Tx||_Y \leq ||T||||x||_X\]
			\item If $Y$ is a Banach space then $L(X,Y)$ is a Banach space.
		\end{itemize}
	\end{theorem}
	\begin{definition}
		For a normed space $(X,||\cdot||_\infty)$ we define the dual space $X' = L(X,\mathbb{K})$. 
	\end{definition}
	\begin{remark}
		$X'$ is a Banach space.
	\end{remark}
	\subsection{Neumann series}
	\begin{lemma}
		Let $X,Y,Z$ be three normed spaces. Let $T \in L(X,Y)$ and $S \in L(Y,Z)$. Then $S\circ T \in L(X,Z)$ and \[||S\circ T||\leq ||S|| ||T||\] 
	\end{lemma}
	Let $T:X\to Y$ be linear, bounded and bijective. Then $\exists T^-1: Y \to X$ linear. 
	\begin{definition}
		Let $X,Y$ be normed spaces. \begin{enumerate}
			\item $T \in L(X,Y)$ is bijective such that $T^-1 \in L(Y,X)$ then $T$ is called an isomorphism
			\item $X,Y$ are called isomorph if there is $T: X \to Y$ isomorphism.
			\item $T \in L(X,Y)$ is called an Isometry if $||Tx|| = ||x||$.
			\item $X,Y$ are called isometric isomorph if $\exists T \in L(X,Y)$ an isomorphism that is also an isometry.
		\end{enumerate}
	\end{definition} 
	\begin{remark}
		The identity $I_x: X \to X$ with $x \mapsto x$ is in $L(X)$. Then $T \in L(X)$ is an isomorphism iff $\exists S \in L(X)$ s.t. $S\circ T = I_x$ and $T\circ S = I_x$
	\end{remark}
	Let $T \in L(X)$ s.t $||T|| < 1$. Define $T^0 = I_x$, $T^n = T\circ T^{n-1}$. Obviously $T^n \in L(X)$ for all $n$. Now, \[\left(\sum_{k=0}^{n}T^k \right)_{n\in \mathbb{N}} \subset L(X)\] is a Cauchy sequence w.r.t. the operatornorm. Hence, if $X$ is a Banach-Space, so is $L(X)$ and thus the series converges to a $S \in L(X)$. Furthermore \[\sum_{k=0}^\infty ||T||^k = \frac{1}{1-||T||}\]
	Finally, we can also note that $S = (I_x-T)^{-1}$.
	\begin{theorem}[Neumann series]
		Let $X$ be a Banach-Space, $T \in L(X)$ with $||T|| < 1$ The $I_x - T$ is an isomorphism and \[(I_x - T)^{-1} = \sum_{k=0}^\infty T^k\] is in $L(X)$. This is called the Neumann series.
	\end{theorem}
	\subsection{The dual space of $l^p$}
	We only deal with $1\leq p < \infty$.
	\begin{theorem}
		Let $q \in (1,\infty]$ be s.t. $\frac{1}{p} + \frac{1}{q} = 1$. Then the dualspace $(l^p)'$ is isometric isomorph to $l^q$.
	\end{theorem}
\end{document}