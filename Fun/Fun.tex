\documentclass[a4paper, 12pt]{article}

\usepackage{fullpage}
\usepackage[utf8]{inputenc}
\usepackage[english]{babel}
\usepackage{amsmath,amssymb}
\usepackage[explicit]{titlesec}
\usepackage{ulem}
\usepackage[onehalfspacing]{setspace}
\usepackage{amsthm}

\theoremstyle{plain}
\newtheorem{theorem}{Theorem}[subsection] % reset theorem numbering for each chapter

\theoremstyle{definition}
\newtheorem{definition}[theorem]{Definition} % definition numbers are dependent on theorem numbers
\theoremstyle{lemma}
\newtheorem{lemma}[theorem]{Lemma}

\theoremstyle{remark}
\newtheorem{remark}[theorem]{Remark}

\theoremstyle{corollary}
\newtheorem{corollary}[theorem]{Corollary}

\theoremstyle{example}
\newtheorem{example}[theorem]{Example}

\titleformat{\subsection}
{\small}{\thesubsection}{1em}{\uline{#1}}
\begin{document}
	\begin{titlepage} 
		\title{Fun Summary}
		\clearpage\maketitle
		\thispagestyle{empty}
	\end{titlepage}
	\tableofcontents
	\newpage
	\section{metric spaces}
	\label{sec: metric spaces}
	\subsection{metric spaces}
	\begin{definition}
		A \underline{metric space} is a non-empty set $X$ together with a map \[d: X \times X \to \mathbb{R}\]
		\[(x,y) \mapsto d(x,y)\]
		such that \begin{enumerate}
			\item $d(x,y) = 0$ iff $x = y$
			\item $d(x,y) = d(y,x)$
			\item $d(x,z) \leq d(x,y) + d(y,z) \; \forall x,y,z \in X$
		\end{enumerate}
	\end{definition}
	
	\begin{remark}($d$ admits only positive values)\\
		\[0 = d(x,x) \leq d(x,y) + d(y,x) = 2d(x,y)\]
	\end{remark}
	
	\begin{example}
		\begin{enumerate}
			\item $d_2(x,y) = ||x-y||_2$
			\item $d(x,y) = \begin{cases}
				0 \; \text{ if } x = y\\
				1 \; \text{ else}
			\end{cases}$
		\end{enumerate}
	\end{example}
	
	\begin{definition} (convergence)\\
		A sequence $(x_n)_{n \in \mathbb{N}}$ in a metric space $(X,d)$ is said to be convergent to $x \in X$ if \[x_n \to x \text{ in } (X,d)\] or \[\lim\limits_{n \to \infty} x_n = x \text{ in } (x,d)\]
	\end{definition}
	
	\subsection{Topology in metric spaces}
	Let $(X,d)$ be a metric space.
	\begin{definition}
		\begin{enumerate}
			\item an open ball is defined by \[B_r(x) = \{y \in X: \; d(x,y) < r\}\]
			\item $O \subset X$ is called open if $\forall y \in O$ there is $r > 0$ such that $B_r(y) \subset O$
			\item $A \subset X$ is closed if $X \setminus A$ is open.
		\end{enumerate}
	\end{definition}
	
	\begin{theorem} (metric spaces are topological spaces)\\
		Let $\mathcal{T}$ be the set of open subsets of $X$. Then \begin{enumerate}
			\item $\varnothing, X \in \mathcal{T}$
			\item if $U,V \in \mathcal{T}$, then $U \cup V \in \mathcal{T}$
			\item if $\{U_i\}_{i \in I} \subset \mathcal{T}$, then $\bigcup_{i \in I} \in \mathcal{T}$
		\end{enumerate}
	\end{theorem}
	
	\begin{remark}
		\begin{enumerate}
			\item $\varnothing, X$ are closed
			\item finite union of closed sets is closed
			\item arbitrary intersections of closed sets is closed
		\end{enumerate}
	\end{remark}
	
	\begin{lemma}
		$A \subset X$ is closed iff $\forall$ convergent sequences $(x_n)_{n \in \mathbb{N}} \subset A$ the limit point is in $A$.
	\end{lemma}
	
	\begin{definition}
		For $M \subset X$ we define \[\overline{M} = \bigcap\limits_{A \supset M, \; A \text{ closed}}\] as the closure of $M$ and \[\overset{°}{M} = \bigcup\limits_{O \subset M, \; O \text{ open}}\] as the interior of $M$.\\
		$\partial M = \overline{M} \setminus \overset{°}{M}$ is the boundary of $M$
	\end{definition}
	
	\noindent\underline{Attention:}\\
	Define the closed ball as $\overline{B}_r(a) = \{y \in X: \; d(y,a) \leq r\}$. Then in general $\overline{B_r(a)} \neq \overline{B}_r(a)$.\\
	Example: Take $X \neq \varnothing$ and the trivial metric $d$. Then \[B_1(a) = \{a\} = \overline{B_1(a)}\] but $\overline{B}_1(a) = X$.
	
	\subsection{separability and completion}
	Let $(X,d)$ be a metric space.
	\begin{definition}
		\begin{enumerate}
			\item $M \subset X$ is called dense in $X$ if $\overline{M} = X$.
			\item $X$ is called separable if $X$ has a countable dense subset.
		\end{enumerate}
	\end{definition}
	
	\begin{remark}
		$M$ is dens in $X$ iff \[\forall x \in X \; \forall \varepsilon > 0 \; \exists y \in M \text{ s.t. } d(x,y) < \varepsilon\]
	\end{remark}
	
	\begin{definition}
		\begin{enumerate}
			\item $(x_n)_{n \in \mathbb{N}} \subset X$ is called a Cauchy sequence if \[\forall \varepsilon > 0 \; \exists N \in \mathbb{N} \text{ s.t. } m,n > N \text{ implies } d(x_n,x_m) < \varepsilon\]
			\item A metric space in which all Cauchy sequences converge is called complete.
		\end{enumerate}
	\end{definition}
	
	\begin{example}
		\begin{enumerate}
			\item $(C^0([a,b], \mathbb{R}), d_\infty)$ with $d_\infty(f,g) = \max\limits_{x \in [a,b]} \left|f(x)-g(x)\right|$ is complete.
			\item $(\mathbb{R}^n, d_2)$ with $d_2(x,y) = ||x-y||_2$ is complete.
		\end{enumerate}
	\end{example}
	
	\begin{lemma}
		Let $(X,d)$ be a complete metric space and $\varnothing \neq A \subset X$. Then $(A,d)$ is complete iff $A$ is closed.
	\end{lemma}
	
	\begin{definition}
		$A \subset X$ is called bounded if its diameter \[diam(A) = \sup\{d(x,y): \; x,y \in A\}\] is finite.
	\end{definition}
	
	\begin{theorem}
		$(X,d)$ is complete iff $\forall (F_n)_{n \in \mathbb{N}}$ sequences of closed subsets such that $F_{n+1} \subset F_n$ and $diam(F_n) \to 0$ then \[\exists ! x_0 \in X \text{ s.t.} \bigcap\limits_{n \in \mathbb{N} F_n = \{x_0\}}\]
	\end{theorem}
	
	\subsection{Continuity}
	\begin{definition}
		Let $(X,d_x), (Y, d_y)$ be metric spaces and $f: X \to Y$. $f$ is continuous in $x_0$ if \[\forall \varepsilon > 0 \; \exists \delta > 0 \text{ s.t. } \forall x \in X \; d_x(x,x_0) < \delta \text{ implies } d_y(f(x),f(x_0)) < \varepsilon\]
		With sequences: \[\forall (x_n)_{n\in \mathbb{N}} \subset X \; x_n \to x_0 \text{ in } (X,d_x) \text{ if it holds } (f(x_n))_{n \in \mathbb{N}} \subset Y, \, f(x_n) \to f(x_0) \text{ in } (Y,d_y)\]
	\end{definition}
	$f$ is continuous if $f$ is continuous in $x_0$ for all $x_0 \in X$.\\
	In other words $f$ is continuous if for all $O \subset Y$ open (closed) $f^{-1}(O)$ is open (closed) in $X$.\\
	\underline{Special case}: $f$ is Lipschitz continuous if $\exists L > 0$ s.t. \[d_y(f(x),f(y)) \leq Ld_x(x,y) \; \forall x,y \in X\]
	$f$ is an isometric if $\forall x,y \in X$ it holds that $d_Y(f(y),f(x)) = d_x(x,y)$.
	\subsection{Compact sets}
	\begin{definition}
		Let $(X,d)$ be a metric space and $A \subset X$. 
		\begin{enumerate}
			\item an open cover of $A$ is a collection $\{U_i\}_{i \in I}$ where $I\neq \varnothing$ is an arbitrary index set of open subsets of $X$ s.t. $A \subset \bigcup\limits_{i \in I} U_i$.
			\item $A$ is compact if every open cover of $A$ contains a finite subcover i.e. there is $N \in \mathbb{N}$ and indices $i_1,...,i_N$ such that \[A \subset U_1 \cup ... \cup U_N\]
			\item $A$ is sequentially compact if every sequence in $A$ has a convergence subsequence in $A$.
			\item $A$ is called precompact or totally bounded if $\forall \varepsilon > 0 \; \exists N \in \mathbb{N}$ and $\exists x_1,...,x_N \in X$ such that $A \subset \bigcup_{i = 1}^N B_\varepsilon(x_i)$.
		\end{enumerate}
	\end{definition}
	
	\begin{theorem}
		Let $(X,d)$ be a metric scape and $A \subset X$. The following are equivalent:
		\begin{enumerate}
			\item $A$ is compact
			\item $A$ is sequentially compact
			\item $(A,d)$ is complete and $A$ is precompact.
		\end{enumerate}
	\end{theorem}
	
	\begin{remark}
		If $A$ is precompact, then $\overline{A}$ is precompact. Further, if $(X,d)$ is complete and $A\subset X$ then $A$ is precompact $\Leftrightarrow$ $\overline{A}$ is compact.
	\end{remark}
	
	Recall: $A$ compact $\Rightarrow$ bounded and closed and $f: X \to Y$ continuous with $A \subset X$ compact, then $f(A)$ is compact as well. Further, if $f:A \to \mathbb{R}$ is continuous and $A$ is compact, then \[\exists x_1, x_2 \in A \; \text{ s.t. } f(x_1) \leq f(x) \leq f(x_2) \; \forall x \in A\]
	Theorem of Heine-Borel: $A \subset \mathbb{R}^n$ is compact iff $A$ is closed and bounded.
	\subsection{Theorem of Baire}
	\begin{theorem}
		\label{th: baire}
		Let $(X,d)$ be a complete metric space and $\forall n \in \mathbb{N}$ consider $U_n \subset X$ open and dense. Then \[\bigcap\limits_{n \in \mathbb{N}} U_n\] is dense in $X$.
	\end{theorem}
	
	\begin{remark}
		\begin{enumerate}
			\item Completeness is in general necessary. Consider $(\mathbb{Q}, d)$ and $d(x,y) = \left|x-y\right|$. Define a sequence $x_n$ such that $\mathbb{Q} = \{x_n \; n \in \mathbb{N}\}$. Take $U_n = \mathbb{Q} \setminus \{x_n\}$ which is open and dense. Then \[\bigcap\limits_{n \in \mathbb{N}} U_n = \varnothing\]
		\end{enumerate}
	\end{remark}
	
	\begin{corollary}
		Let $(X,d)$ be a complete metric space. Let $\forall n \in \mathbb{N}$, $A_n \subset X$ be closed and \[X = \bigcup\limits_{n \in \mathbb{N}} A_n\] Then $\exists N \in \mathbb{N}$ s.t. $A_N$ has an interior point.
	\end{corollary}
	
	\begin{remark}
		Theorem \ref{th: baire} is also called \underline{Baire category theory}.\\
		\begin{itemize}
			\item In a metric space $(X,d)$ $A \subset X$ is called nowhere dense if $\overline{A}$ has no interior points.
			\item $A$ is called of first category if $\exists (M_n)_{n \in \mathbb{N}}$ where $M_n \subset A$ nowhere dense s.t. $A = \bigcup_{n \in \mathbb{N}} M_n$
			\item $A$ is called of second category if it is not of first category
		\end{itemize}
		Hence the theorem of Baire implies that every complete metric space is of second category.
	\end{remark}
	
	\section{Normal spaces and Banach spaces}
	Let $X$ be a $\mathbb{K}$-vector space where $\mathbb{K} = \mathbb{R}$ or $\mathbb{C}$.
	\subsection{definitions}
	\begin{definition}
		A map $||\cdot||: X \to \mathbb{R}$ is called a norm on $X$ if \begin{enumerate}
			\item $\forall x \in X$, $||x||\geq 0$ and $||x|| = 0$ iff $x = 0$
			\item $\forall \lambda \in \mathbb{K}$ and $\forall x \in X$ it holds that $||\lambda x|| = \left|\lambda\right|\cdot ||x||$
			\item $\forall x, y \in X$ it holds $||x+y|| \leq ||x|| + ||y||$
		\end{enumerate}
		The pair $(X,||\cdot||)$ is called an normed space.\\
		
		\noindent $p:X \to \mathbb{R}$ is called a seminorm if $p(x) \geq 0$ $\forall x \in X$ and 2. and 3. are also satisfied.
	\end{definition}
	
	\begin{example}
		\begin{enumerate}
			\item $C^0([0,1]; \mathbb{R})$ with $||f||_\infty = \max\limits_{x \in [0,1]} \left|f(x)\right|$
			\item more general for a compact metric space $K$: $C^0(K,\mathbb{R})$ with $||f||_\infty = \max\limits_{x \in K} \left|f(x)\right|$
			\item $C^1([0,1]; \mathbb{R})$ with $p(f) = \max\limits_{x \in [0,1]} \left|f'(x)\right|$
			\item $\Omega \subset \mathbb{R}^n$ measurable. $L^1(\Omega) = \{f: \Omega \to \mathbb{R}: \; f \text{ integrable }\}$ with \[p: L^(\Omega) \to \mathbb{R}: \; p(f) = \int_{\Omega} \left|f(x)\right|dx\] then $p$ is a seminorm.
		\end{enumerate}
	\end{example}
	\begin{remark}
		Any normed space is a metric space via \[d(x,y) = ||x-y||\] All concepts from chapter \ref{sec: metric spaces} apply.
	\end{remark}
	
	\begin{lemma}
		Let $(X,||\cdot||)$ be a normed space. Then $X$ is called separable iff $\exists A \subset X$ countable such that s.t. $\overline{span\{A\}} = X$ where $span\{A\} = \{\sum_{i=1}^{n} \lambda_i x_i$\} with $n \in \mathbb{N}$, $\lambda_i \in K$ and $x_i \in A$. Here the colusre is defined w.r.t the norm.
	\end{lemma}
	
	\begin{definition}
		A complete normed space is called a Banach space.
	\end{definition}
	
	\subsection{Example: $l^p$-spaces}
	We consider the vector space $\mathbb{K}^\mathbb{N}$ of sequences in in $\mathbb{K}$. Let $x = (x_n)_{n \in \mathbb{N}}$ and $y = (y_n)_{n \in \mathbb{N}}$. Define $x+y = (x_n+y_n)_{n \in \mathbb{N}}$ and $\lambda x = (\lambda x_n)_{n \in \mathbb{N}}$.\\
	For $x \in \mathbb{K}^\mathbb{N}$ define \[||x||_{l^p} = \left(\sum_{n=1}^\infty \left|x\right|^p\right)^{1/p}\] for $1\leq p < \infty$ and \[||x||_{l^\infty} = \sup\limits_{n \in \mathbb{N}} \left|x_n\right|\] else.\\
	Define $l^p = \{x = (x_n)_{n \in \mathbb{N}}: ||x||_{l^p} < \infty\}$ for $1 \leq p \leq \infty$. We find that $l^p$ is a subspace of $\mathbb{K}^\mathbb{N}$ and $l^p$ is a normed space (for the triangle inequality use the Hölder inequality).
	\begin{theorem}
		For $1 \leq p \leq \infty$ $l^p$ is a Banach space.
	\end{theorem}
	\begin{lemma}
		For finite $p$, $l^p$ is separable while $l^\infty$ is not.
	\end{lemma}
	
	\subsection{Finite dimensional normed spaces}
	Let $X$ be a vector space over $\mathbb{K}$. $\exists e_1,...,e_n \in X$ s.t. \[\forall x \in X; \; \exists \lambda_1,...,\lambda_n \in \mathbb{K}: \; x = \sum_{i=1}^n \lambda_i x_i\]
	For $p \in [1,\infty)$ we define \[||x||_p = \left(\sum_{i=1}^n \left|\lambda_i\right|^p\right)^{1/p}\] and for $p = \infty$ \[||x||_\infty = \max\limits_{1 \leq i \leq n} \left|\lambda_i\right|\]
	
	\begin{definition}
		Two norms are equivalent in that \[\alpha ||\cdot ||_1 \leq ||\cdot ||_2 \leq \beta ||\cdot ||_1\]
	\end{definition}
	
	\begin{theorem}
		In a finite dimensional space, all norms are equivalent.
	\end{theorem}
	
	\begin{theorem}
		Finite dimensional normed spaces are Banach spaces.
	\end{theorem}
	
	\subsection{On the closure of $\overline{B_1(0)}$}
	\begin{lemma}[Lemma of Riesz, Lemma of the almost orthogonal element]
		Let $X$ be a normed space. $U \subset X$ a closed subspace of $X$ s.t. $U \neq X$. Then $\forall \lambda \in (0,1) \exists x_\lambda \in X$ s.t. $||x_\lambda|| = 1$ and $dist(x_\lambda, U) \geq \lambda$.
	\end{lemma}
	\begin{theorem}
		In a normed space $X$, $\overline{B_1(0)}$ is compact iff $X$ is finite dimensional.
	\end{theorem}
	
	\section{A question from approximation theory}
	\subsection{Theorem of Stone-Weierstrass}
	Let $X$ be a compact metric space. Then $(C^0(X), \mathbb{K}), ||\cdot ||_\infty$, where $||f||_\infty = \max\limits_{x \in X} \left|f(x)\right|$ is a Banach space.\\
	Which property of $A \subset C^0(X, \mathbb{K})$ ensures that $A$ is dense.
	\begin{definition}
		$A \subset C^0(X, \mathbb{K})$ is called subalgebra, if $\forall f,g, \in A$ \begin{enumerate}
			\item $\lambda f + \mu g \in A$ (subspace)
			\item $f\cdot g \in A$
		\end{enumerate}
	\end{definition}  
	\begin{example}
		\begin{itemize}
			\item $\{p: [0,1] \to \mathbb{R}\}$ is a subalgebra of $C^0([0,1]; \mathbb{R})$. 
			\item $\{f: [-1,1] \to \mathbb{R}; f\text{ continuous and even}\}$ is a subalgebra.
		\end{itemize}
	\end{example}
	\begin{remark}
		If $A$ is a subalgebra, then $\overline{A}$ is also a subalgebra.
	\end{remark}
	\begin{definition}
		Let $A \subset C^0(X)$ be a subalgebra. \begin{enumerate}
			\item $A$ is called unital if $1 \in A$
			\item $A$ separates point if $x,y \in X$, $x \neq y$, $\exists f \in A$ s.t. $f(x) \neq f(y)$.
			\item (if $\mathbb{K} = \mathbb{C}$) $A$ is stable under conjuguation if from $f \in A$ we conclude that also $\overline{f} \in A$.
		\end{enumerate}
	\end{definition}
	\begin{remark}
		If $A$ is unital then all constant functions are in $A$.
	\end{remark}
	\begin{lemma}
		Consider $f: [-1,1] \to \mathbb{R}$ where $f(x) = \left|x\right|$. Then $\exists$ sequence of polynomials $(p_n)_{n \in \mathbb{N}}$ s.t. \[p_n \to f\] uniformly in $[-1,1]$.
	\end{lemma}
	\begin{lemma}
		Let $A \subset C^0(X, \mathbb{R})$ be a unital subalgebra. Then \begin{enumerate}
			\item if $f \in A$ then $\left|f\right| \in \overline{A}$.
			\item if $f,g \in A$ then $\max\{f,g\} \in \overline{A}$ and $\min\{f,g\} \in \overline{A}$
		\end{enumerate}
	\end{lemma}
	
	\begin{theorem}[Stone-Weierstrass]
		Let $A$ be a compact metric space. $A \subset C^0(X,\mathbb{K})$ is a unital subalgebra that separates points and if $\mathbb{K} = \mathbb{C}$ is stable under conjugation, then $A$ is dense in $C^0(X,\mathbb{K})$ w.r.t $||\cdot||_\infty$.
	\end{theorem}
	\subsection{Applications}
	\begin{theorem}[Theorem of Weierstraß]
		Let $[a,b]$ be a compact interval in $\mathbb{R}$, $f:[a,b] \to \mathbb{R}$ be a continuous function and $\varepsilon>0$. Then $\exists p: [a,b] \to \mathbb{R}$ a polynomial s.t. \[||p-f||_\infty = \sup_{x \in [a,b]} \left|p(x)-f(x)\right| < \varepsilon\]
	\end{theorem}
	\begin{definition}
		A function $f:\mathbb{R} \to \mathbb{C}$ is periodic if \[f(x+t) = f(x)\] for a $t \in \mathbb{R}$ and all $x \in \mathbb{R}$.
	\end{definition}
	\begin{remark}
		If $f$ is periodic with period $t$ then $\tilde{f}: \mathbb{R} \to \mathbb{C}$ where $\tilde{f}(x) = f(t\frac{x}{2\pi})$ is periodic of period $2\pi$.\\
		Consider $C_{2\pi}^0(\mathbb{R}, \mathbb{C})$ the space of continuous $2\pi$-periodic functions. We consider the span of $\{e^{ikx} = \cos(kx)+i\sin(kx),k \in \mathbb{Z}\}$.
	\end{remark}
	\begin{definition}
		A trigonometric polynomial is a function $f:\mathbb{R} \to \mathbb{C}$ \[f(x) = \sum_{k=-N}^{N} c_k\cdot e^{ikx}\] with $c_k \in \mathbb{C}$
	\end{definition}
	\begin{theorem}[Approximation of periodic functions]
		Trigonometric polynomials are dense in $(C_{2\pi}^0(\mathbb{R},\mathbb{C}),||\cdot||_\infty)$ 
	\end{theorem}
	\underline{Application to neural networks}\\
	The simplest case of a neural network has $d$ inputs $x_1,...,x_d$ and one output $Z$ called a \textit{feed forward} network. Each input influences the output and $x_i$ might have a weight $\alpha_i$ associated to it. The output is a function in $x=(x_1,...,x_d)$ and the weights $\alpha= (\alpha_1,...,\alpha_d)$. For instance, the output is often of the form \[Z = \sum_{i=1}^d \alpha_ix_i + b\] where $b$ is the bias of the network. To make the network slightly stronger, we add a intermediate layer $y = (y_1,...,y_r)$ where each $x_i$ is connected to each $y_j$ with the associated weight $\gamma_{i,j}$. The $y$ layer (often called activation) is the connected to the output $Z$ as above with weights $\alpha_j$. We introduce the realtion \[y_j = \Phi(\sum_{i=1}^d \gamma_{j,i}x_i+b)\] for a measurable function $\Phi$. Lastly, the output is then given by \[Z = \sum_{j=1}^r \alpha_j y_j\]
	\begin{definition}
		\begin{enumerate}
			\item $A^d = \{a: \mathbb{R}^d \to \mathbb{R}: \; a(x9 = w^Tx+b)\}$ where $w \in \mathbb{R}^d$ and $b \in \mathbb{R}$.
			\item given $\Phi: \mathbb{R} \to \mathbb{R}$ measurable $d \in \mathbb{N}$ define $\Sigma^d(\Phi) = \{f: \mathbb{R}^d \to \mathbb{R}: \; f(x) = \sum_{j=1}^N \alpha_j \Phi(a_j(x)) \text{ wtih } N \in \mathbb{N}, \alpha_j \in \mathbb{R}, a_j \in A^d\}$ as the set of single hidden layer feed forward networks.
			\item A squashing function is a measurable non-decreasing function $\Phi: \mathbb{R} \to \mathbb{R}$ s.t. $\lim_{x \to -\infty} \Phi(x) = 0$ and $\lim_{x \to \infty} \Phi(x) = 1$.
		\end{enumerate}
	\end{definition}
	\begin{theorem}[Universal Approximation theorem of Hornik-Stinchcombe-White]
		Let $\Phi$ we a squashing function $K\subset \mathbb{R}^d$ compact $f:K \to \mathbb{R}$ continuous and $\varepsilon > 0$. Then $\exists g \in \Sigma^d(\Phi)$ s.t. \[\sup_{x \in K} \left|f(x)-g(x)\right| < \varepsilon\]
	\end{theorem}
	
	\section{Continuous linear maps}
	$(X,||\cdot ||_X), (Y,||\cdot ||:Y)$ are $K$-Vector spaces with $K = \mathbb{R}$ or $K = \mathbb{C}$. $T:X \to Y$ is called linear if \[T(\lambda_1x_1 + \lambda_2 x_2) = \lambda_1 T(x_1) + \lambda_2 T(x_2)\]
	\subsection{Continuity of linear maps}
	\begin{definition}
		LEt $T:X\to Y$ be linear. Then $T$ is bounded if $\exists C>0$ s.t. \[||Tx||_Y \leq C||x||_X \; \forall x \in X\]
		or equivalently \[\sup_{x \in X\setminus \{0\}} \frac{||Tx||_Y}{||x||_X} \leq C\] which is also equivalent to \[\sup_{x \in X, ||x||_X=1} ||Tx||_Y \leq C\]
	\end{definition}
	\begin{theorem}
		For $T: X \to Y$ linear, the following are equivalent: \begin{enumerate}
			\item $T$ is continuous 
			\item $T$ is continuous in 0
			\item $t$ is bounded
		\end{enumerate}
	\end{theorem}
	\begin{lemma}
		Let $X$ have infinite dimension. Then $\exists T: X \to \mathbb{K}$ linear and not bounded.
	\end{lemma}
	\begin{definition}
		Define $L(X,Y)$ as the set of linear continuous ($\Leftrightarrow$ bounded) maps from $X$ to $Y$. With the usual addition $((T_1+T_2)(x) = T_1(X) + T_2(x))$ and the scalar multiplication $((\lambda(T)(x)) = \lambda T(x))$ this is a vector space.\\
		If $X=Y$ we write $L(X)$. For $T \in L(X,Y)$ \[\ker T = \{x \in X: Tx = 0\}\] and \[\Im(T) = \{y \in Y: \exists x\in X: \; Tx = y\}\]
	\end{definition}
	\subsection{Operatornorm and dual space}
	\begin{theorem}
		Let $X \neq \{0\}$. \begin{itemize}
			\item $L(X,Y)$ with the operatornorm $||T|| = \sup_{x \in X\setminus \{0\}} \frac{||Tx||_Y}{||x||_X} = \sup_{x \in X, ||x||_X=1} ||Tx||_Y$ is a normed space. We have \[||Tx||_Y \leq ||T||||x||_X\]
			\item If $Y$ is a Banach space then $L(X,Y)$ is a Banach space.
		\end{itemize}
	\end{theorem}
	\begin{definition}
		For a normed space $(X,||\cdot||_\infty)$ we define the dual space $X' = L(X,\mathbb{K})$. 
	\end{definition}
	\begin{remark}
		$X'$ is a Banach space.
	\end{remark}
	\subsection{Neumann series}
	\begin{lemma}
		Let $X,Y,Z$ be three normed spaces. Let $T \in L(X,Y)$ and $S \in L(Y,Z)$. Then $S\circ T \in L(X,Z)$ and \[||S\circ T||\leq ||S|| ||T||\] 
	\end{lemma}
	Let $T:X\to Y$ be linear, bounded and bijective. Then $\exists T^-1: Y \to X$ linear. 
	\begin{definition}
		Let $X,Y$ be normed spaces. \begin{enumerate}
			\item $T \in L(X,Y)$ is bijective such that $T^-1 \in L(Y,X)$ then $T$ is called an isomorphism
			\item $X,Y$ are called isomorph if there is $T: X \to Y$ isomorphism.
			\item $T \in L(X,Y)$ is called an Isometry if $||Tx|| = ||x||$.
			\item $X,Y$ are called isometric isomorph if $\exists T \in L(X,Y)$ an isomorphism that is also an isometry.
		\end{enumerate}
	\end{definition} 
	\begin{remark}
		The identity $I_x: X \to X$ with $x \mapsto x$ is in $L(X)$. Then $T \in L(X)$ is an isomorphism iff $\exists S \in L(X)$ s.t. $S\circ T = I_x$ and $T\circ S = I_x$
	\end{remark}
	Let $T \in L(X)$ s.t $||T|| < 1$. Define $T^0 = I_x$, $T^n = T\circ T^{n-1}$. Obviously $T^n \in L(X)$ for all $n$. Now, \[\left(\sum_{k=0}^{n}T^k \right)_{n\in \mathbb{N}} \subset L(X)\] is a Cauchy sequence w.r.t. the operatornorm. Hence, if $X$ is a Banach-Space, so is $L(X)$ and thus the series converges to a $S \in L(X)$. Furthermore \[\sum_{k=0}^\infty ||T||^k = \frac{1}{1-||T||}\]
	Finally, we can also note that $S = (I_x-T)^{-1}$.
	\begin{theorem}[Neumann series]
		Let $X$ be a Banach-Space, $T \in L(X)$ with $||T|| < 1$ The $I_x - T$ is an isomorphism and \[(I_x - T)^{-1} = \sum_{k=0}^\infty T^k\] is in $L(X)$. This is called the Neumann series.
	\end{theorem}
	\subsection{The dual space of $l^p$}
	We only deal with $1\leq p < \infty$.
	\begin{theorem}
		Let $q \in (1,\infty]$ be s.t. $\frac{1}{p} + \frac{1}{q} = 1$. Then the dualspace $(l^p)'$ is isometric isomorph to $l^q$.
	\end{theorem}
	\section{Theorem of Hahn-Banach}
	Let $X$ be a vector space, $X \neq \{0\}$ over $\mathbb{K} = \mathbb{R}$.
	\subsection{Extension Theorem}
	Given $U\subset X$ subspace, $l: U \to \mathbb{R}$ linear, is there $L: X \to \mathbb{R}$ linear such that $L|_U = l$? For this we need Zorn's Lemma:
	\begin{definition}
		Let $M\neq \varnothing$ be a set and $\leq$ be a partial order on $M$, i.e. $\leq$ satisfies
		\begin{enumerate}
			\item reflexiv: $x \leq x$ $\forall x \in M$
			\item antisymmetric: $x \leq y$ and $y \leq x \Rightarrow x = y$
			\item transitivity $x \leq y, y \leq z \Rightarrow x \leq z$
		\end{enumerate}
		Then
		\begin{itemize}
			\item $A \subset M$ is called chain of totally ordered if $\forall x,y \in A$ either $x \leq y$ or $y \leq x$
			\item $b \in M$ is an upper bound for a chain $A$ if $a \leq b$ for all $a \in A$
			\item $m \in M$ is called maximal element if \[\forall x \in M \text{ s.t. } m \leq x \Rightarrow x = m\]
		\end{itemize}
	\end{definition}
	\begin{lemma} [Zorn]
		Let $M\neq \varnothing$ and $\leq$ be a partial order on $M$. If every chain in $M$ has an upper bound in $M$, then there is a maximal element.
	\end{lemma}
	\begin{definition}
		LEt $X$ be a vector space. $p: X \to \mathbb{R}$ is called sublinear if \begin{enumerate}
			\item $p(\lambda x) = \lambda p(x)$ for all $x \in X, \lambda \geq 0$
			\item $p(x+y) \leq p(x)+p(y)$ for all $x,y \in X$
		\end{enumerate}
	\end{definition}
	\begin{theorem}[Extension theorem of Hahn-Banach]
		Let $X$ be a vecorspace over $\mathbb{R}$, $U\subset X$ a subspace and $U\neq X$. Let $p:X \to \mathbb{R}$ be a subspace $l: U \to \mathbb{R}$ be linear s.t. $l(x) \leq p(x) \; \forall x \in U$. Then $\exists L:X \to \mathbb{R}$ linear s.t. $L(x) \leq p(x) \; \forall x \in X$ and $L(x) = l(x) \; \forall x \in U$. $L$ is called extension of $l$.
	\end{theorem}
	\underline{Consequences for normed spaces}
	\begin{theorem}
		Let $(X,||\cdot ||_X)$, $U \subset X$ a subspace fo $X$, with $U \neq X$. Let $u' \in U' =  L(U,\mathbb{R})$. Then $\exists x' \in X'$ s.t. $||x'||_{X'} = ||u'||_{U'}$ such that $x'(u) = u'(u) \; \forall u \in U$.
	\end{theorem}
	\begin{corollary}
		Let $(X,||\cdot||_X)$, $U\subset X$ be a subspace of $X$ and $x_0 \in X$ s.t. $dist(x_0,U)>0$. Then $\exists x' \in X'$ s.t. $x'|_U = 0$ $\forall u \in U$ and $x'(x_0) = dist(x_0,U)$ with $||x'||_{X'} = 1$. 
	\end{corollary}
	\begin{corollary}
		Let $X,||\cdot||_X)$ and $x_0 \in X$. \begin{enumerate}
			\item if $x_0 \neq 0$ then $\exists F \in X'$ with $||F||_{X'}=1$ and $F(x_0) = ||x_0||_X$ In particular, for $x \in X$ \[||x||_X = \sup_{F\in X', ||F||_{X'}=1}\left|F(x)\right|\]
			\item If $F(x_0) = 0$ for all $F\in X'$, then $x_0 = 0$. In particular, $X'$ separates points of $X$.
			\item $U \subset X$ subspace. Then $U$ is dense in $X$ iff if for $x'\in X'$ s.t. $x'_{|_U} = 0$ it follows $x' = 0$.
		\end{enumerate}
	\end{corollary}
	\subsection{Separation Theorems}
	\begin{definition}
		Let $X$ be a vectorspace over $\mathbb{R}$. $A\subset X$ is called convex, if \[\forall x,y \in A, \; \lambda x + (1-\lambda) y \in A, \; \forall \lambda \in [0,1]\]
	\end{definition}
	\begin{lemma}
		Let $C\subset X$ open and convex with $O \in C$. Define $p_C: X \to \mathbb{R}$ such that $p_C(x) = \inf\{\lambda > 0 \text{ s.t. } \frac{x}{\lambda} \in C\}$. This is called the Minkowski functional.\\
		Then $p_C$ is sublinear and $C = \{x \in X: p_C(x) < 1\}$.
	\end{lemma}
	\begin{lemma}
		Let $(X,||\cdot ||)$ be a normed space and $A\subset X$ be convex and open, $A \neq \varnothing$ and $x_0 \in X\setminus A$, then $\exists F \in X'$ s.t. \[F(x) < F(x_0) \; \forall x \in A\]
	\end{lemma}
	\begin{definition}
		Let $X\neq \{0\}$ be a $\mathbb{R}$-vectorspace. \begin{enumerate}
			\item $H = \{x \in X: \; f(x) = \alpha\}$ with $f: X \to \mathbb{R}$ linear, $\alpha \in \mathbb{R}$
			\item $A,B \subset X$ are separated by an affine hyperplane $H$ if $H = \{f = \alpha\}$ and $f(a) \leq \alpha \leq f(b)$ $\forall a \in A\; \forall b \in B$.
			\item $A,B \subset X$ are strictly separated by an affine Hyperplane $H$ if $\exists \varepsilon > 0$ s.t. $f(a) + \varepsilon \leq \alpha \leq f(b) - \varepsilon$.
		\end{enumerate}
	\end{definition}
	\begin{theorem}[Separation Theorem of Hahn-Banach]
		Let $(X, ||\cdot ||)$, $A,B$ convex, $A \neq \varnothing$, $b \neq \varnothing$, $A\cap B = \varnothing$ and $A$ or $B$ should be open.. Then $\exists F \in X'$ and $\delta \in \mathbb{R}$ s.t. \[F(a) \leq \delta \leq F(b) \; \forall a \in A, b \in B\]
	\end{theorem}
	\begin{theorem}
		Let $(x,||\cdot ||)$, $A, B$ convex subsets $A \neq \varnothing, B \neq \varnothing$, $A \cap B = \varnothing$. Let $A$ be closed and $B$ be compact. Then $\exists F \in X', \exists \varepsilon > 0$ s.t. $F(a) + \varepsilon \leq F(b) - \varepsilon$ $\forall a \in A, b \in B$.
	\end{theorem}
	\section{Hilbert Spaces}
	Let $X$ be a vector space over $\mathbb{K}$ where $\mathbb{K} = \mathbb{R}$ or $\mathbb{C}$.
	\subsection{Inner product space}
	\begin{definition}
		A map $\langle \cdot , \cdot \rangle$: $X\times X \to \mathbb{K}$ is an inner product on $X$, if \begin{enumerate}
			\item $\langle x_1+x_2, y \rangle = \langle x_1,y\rangle + \langle x_2,y\rangle$
			\item $\langle \lambda x,y\rangle = \lambda \langle x,y\rangle$
			\item $\langle x,y\rangle = \overline{\langle x,y\rangle}$
			\item $\langle x,y\rangle \geq 0$ and $\langle x,x\rangle = 0 \Leftrightarrow x=0$
		\end{enumerate}
		The pair $(X, \langle\cdot,\cdot\rangle)$ is called an inner product space also called a pre-Hilbert-space.\\
		An inner product is a symmetric bilinear form if $\mathbb{K} = \mathbb{R}$ and a sesquilinear form if $\mathbb{K} = \mathbb{R}$.
	\end{definition}
	\begin{theorem} [Cauchy-Schwartz-inequality]
		In an inner product space we have \[\left|\langle x,y\rangle\right| \leq \sqrt{\langle x,x\rangle} \sqrt{\langle y,y\rangle}\]
	\end{theorem}
	\begin{theorem}
		For an inner product space $X$ we define $||\cdot ||: X \to [0,\infty)$ by $||x|| = \sqrt{\langle x,x\rangle}$. This is a norm.
	\end{theorem}
	\begin{definition}
		Let $X$ be an inner product space. Then $x,y \in X$ are orthogonal if $\langle x,y\rangle = 0$. This is written as $x\bot y$.
	\end{definition}
	\begin{corollary}
		If $x\bot y$, then \[||x+y||^2 = ||x||^2 + ||y||^2\]
	\end{corollary}
	\begin{theorem}
		A normed space is an inner product space, iff $\forall x,y \in X$ the norm satisfies  \[||x+y||^2 + ||x-y||^2 = 2||x||^2 + 2||y||^2\]
	\end{theorem}
	\subsection{Hilbert spaces}
	\begin{definition}
		Is an inner product space complete w.r.t. to the induced norm, we call it \underline{Hilbert space}. 
	\end{definition}
	\begin{theorem}[projection theorem]
		Let $X$ be a Hilbert space, $A \subset X$ non-empty, convex and closed. Then $\forall x \in X$ \[\exists ! y \in A \text{ s.t. } ||x-y|| = dist(x,A)\]
		$y$ is called the best approximation or projection of $x$ in $A$.
	\end{theorem}
	\begin{theorem}[Characterisation of the bes approximation]
		Let $X$ be an inner product space, $A\subset X$, $A\neq \varnothing$ and convex and $x\in X$. Then $y$ is the best approximation of $x$ in A iff \[\Re \langle x-y, z-y \rangle \leq 0\; \forall z \in A\]
	\end{theorem}
	\begin{definition}
		Let $X$ be an inner product space, $A\subset X$, then \[A^\bot = \{x\in X: x\bot y \; \forall y \in A\}\] the orthogonal complement of $A$.
	\end{definition}
	\begin{remark}
		$A^\bot$ is a closed subspace. If $(x_n)_{n \in \mathbb{N}} \subset A^\bot$, $x_n \to x$ in $X$, $\forall n \in \mathbb{N}$ we have $\langle x_n,y\rangle = 0 \; \forall y \in A$. Moreover $A\subset (A^\bot)^\bot$.
	\end{remark}
	\begin{theorem}
		Let $X$ be a Hilbert space, $U\subset X$ closed subspace. Then \[\forall x \in X \; \exists ! u \in U \text{ s.t. } ||x-u|| = dist(x,U) = \inf_{z\in U} ||z-u||\]
		We have $x-u \in U^\bot$ and $X = U \oplus U^\bot$, meaning that $x = u+v$, $u\in U, \; v \in U^\bot$ uniquely. The $u$ is called the orthogonal projection of x in $U$.
	\end{theorem}
	\begin{theorem}[Riesz-Fréchet]
		Let $X \neq \{0\}$ be a Hilbert space. $\forall F \in X' \; \exists ! y \in X$ s.t. $F(x) = \langle x,y\rangle$. Moreover, $||F||_{X'} = ||y||_X$. Equivalently \[J: X \to X', \; (Jy)(x) = \langle x,y\rangle\] is a bijective, anti-linear isometry. In particular, if $X'$ is a Hilbert space, the dual is also a Hilbert space.
	\end{theorem}
	\subsection{Orthonormal systems}
	Let $(X\langle\cdot , \cdot \rangle)$ be an inner product space. 
	\begin{definition}
		Let $I \neq \varnothing$ be an index set. A family of vectors $(e_k)_{k\in I} \subset X$ is called an orthonormal system (ONS) if \[\langle e_i,e_j\rangle = \delta_{i,j}\] 
	\end{definition}
	\begin{theorem}[Schmidt Orthogonalisation theorem]
		Let $\{x_i: i \in I\} \subset X, I \subset \mathbb{N}$ be linearly independent vectors. Then $\exists$ ONS $\{e_i: i \in I\}$ s.t. \[span\{x_i: i \in I\} = span\{e_i: i \in I\}\]
	\end{theorem}
	\begin{lemma}[Bessel's inequality]
		Let $\{e_1,...,e_n\}$ be an ONS. $Y = span\{e_1,...,e_n\}$. Then $\forall x \in X$ \[\inf_{y \in Y} ||x-y||^2 = ||x-\sum_{i=1}^n \langle x_i,e_i\rangle||^2 = ||x||^2 - \sum_{i=1}^n \left|\langle x_i, e_i\rangle\right|^2 \geq 0\]
	\end{lemma}
	\begin{definition}
		If $I\subset \mathbb{N}$, $(e_n)_{n\in I}$ ONS, then $\langle x, e_n \angle$ is called the $n$-th Fourier coefficient. of $x$. W.r.t. $(e_n)_{n \in I}$.
	\end{definition}
	\begin{definition}
		An ONS $(e_n)_{n\in \mathbb{N}}, \; I \subset \mathbb{N}$ is called complete in $X$ if \[\overline{span\{e_n: n \in I\}} = X\]
	\end{definition}
	\begin{theorem}
		Any separable Hilbert space $X$ has a complete ONS.
	\end{theorem}
	\begin{lemma}
		Let $X$ be a Hilbert space, $(e_n)_{n \in \mathbb{N}}$ an ONS. Then $\exists y \in X$ s.t. $y = \sum_{n \in \mathbb{N}} \langle x, e_n\rangle e_n$. 
	\end{lemma}
	\begin{theorem}
		Let $X$ be a Hilbert space of infinite dimension, $(e_n)_{n\in \mathbb{N}}$ an orthonormal system. Then the following are equivalent. 
		\begin{enumerate}
			\item $(e_n)_{n\in \mathbb{N}}$ is complete
			\item if $x \in X$ s.t. $\langle x, e_n \rangle = 0 \; \forall n \in \mathbb{N}$, then $x=0$
			\item $\forall x \in X$, $x = \sum_{n=1}^\infty \langle x,e_n\rangle e_n$ (Fourier series of $x$)
			\item $\forall x \in X$, $||x||^2 = \sum_{n=1}^\infty \left|\langle x,e_n\rangle\right|^2$. 
		\end{enumerate}
	\end{theorem}
	\begin{corollary}
		Any separable infinite-dimensional Hilbert space is isometrically isomorphic to $\ell^2$.
	\end{corollary}
	\section{Spectral theorem for self-adjoint compact operators}
	We only deal with Hilbert spaces.
	\subsection{Adjoint in Hilbert spaces}
	Let $(X,\langle\cdot, \cdot \rangle)$, $(Y,\langle\cdot, \cdot \rangle)$, $T \in L(X,Y)$. Let $y \in Y$. Consider the map \[X\ni x \mapsto \langle Tx,y\rangle_Y\] 
	This map is linear and bounded.\[\left|\langle Tx,y\rangle_Y\right| \overset{CS}{\leq} ||Tx||_Y||y||_Y \leq ||T||||x||_X||y||_Y\] Thus it is an element of $X'$. By the theorem of Riesz-Fréchet \[\exists! T^*y \in X \text{ s.t. } \langle x,T^*y\rangle_X= \langle Tx,y\rangle_Y \; \forall x \in X\]
	This defines a map $T^*: Y \to X$ with $y \mapsto T^*y$.
	\begin{definition}
		$T^*$ is the Hilbert space adjoint of $T$. 
	\end{definition}
	\begin{lemma}
		$T^* \in L(Y,X)$ and $||T^*|| = ||T||$. 
	\end{lemma}
	\begin{lemma}
		Let $(X,\langle\cdot ,\cdot \rangle_X)$, $(Y, \langle \cdot , \cdot \rangle_Y)$, $(Z,\langle \cdot, \cdot \rangle_Z)$ be Hilbert spaces. Let $T \in L(X,y)$, $S \in L(Y,Z)$ and $\lambda \in \mathbb{K}$. Then \begin{enumerate}
			\item $(S\circ T)^* = T^*S^*$
			\item $(\lambda T)^* = \overline{\lambda}T^*$
			\item $(T^*)^* = T$ 
		\end{enumerate}
	\end{lemma}
	\begin{definition}
		Let $(X,\langle\cdot ,\cdot \rangle_X)$ be a Hilbert space  and $T\in L(X)$. $T$ is called self-adjoint if $T^* = T$
	\end{definition}
	\begin{lemma}
		\begin{itemize}
			\item If $\mathbb{K} = \mathbb{C}$, $T$ is self-adjoint $\Leftrightarrow \langle Tx,x\rangle_X \in \mathbb{R} \; \forall x \in X$
			\item If $T$ is self-adjoint, then $||T|| = \sup_{x \in X, ||x||_X=1} \left|\langle Tx, x\rangle\right|$ 
		\end{itemize}
	\end{lemma}
	\subsection{compact operators}
	Here $X$, $Y$ can be only Banach spaces and $X,Y \neq \{0\}$.
	\begin{definition}
		$f: X \to Y$ is compact if f maps bounded sets in precompact sets.
	\end{definition}
	\begin{lemma}
		Let $T: X \to Y$ be linear. Then $T$ is compact iff $T(B_1(0))$ is precompact in $Y$.
	\end{lemma}
	\noindent\textbf{Notation:} $K(X,Y) = \{T: X \to Y $ linear and compact$\}$ and $K(X) = K(X,X)$.
	\begin{remark}
		$T \in K(X,Y) \Rightarrow T \in L(X,Y)$.
	\end{remark}
	\begin{lemma}
		\begin{enumerate}
			\item $T \in L(X,Y)$, $S\in L(Y,Z)$. If $T$ or $S$ is compact, then the composition is compact.
			\item $K(X,Y)$ is a closed subspace of $L(X,Y)$. In particular $K(X,Y)$ is a Banach space.
		\end{enumerate}
	\end{lemma}
	\begin{definition}
		\begin{itemize}
			\item Let $H$ be a Hilbert space and $T\in L(X)$. Then $T$ is called self-adjoint if \[\langle Tx,y\rangle = \langle x,Ty\rangle \; \forall x,y \in X\]
			\item Let $X,Y$ be Banach spaces then, $T\in L(X,Y)$ compact $\Leftrightarrow T(B_1(0))$ is precompact.
		\end{itemize}
	\end{definition}
	\begin{lemma}
		$T\in L(X,Y)$ is compact iff $\forall (x_n)_{n\in \mathbb{N}} \subset X$ bounded $(T(x_n))_{n\in \mathbb{N}}$ admits a convergent subsequence.
	\end{lemma}
	\subsection{Spectrum}
	Let $X$ be a Banach space.
	\begin{definition}
		Let $T\in L(X)$. \begin{itemize}
			\item the resolvent set of $T$ is \[\rho(T) = \{\lambda \in \mathbb{K}: (\lambda \cdot Id - T)^{-1}\in L(X)\}\subset \mathbb{K}\]
			while $\sigma(T) = \mathbb{K}\setminus \rho(T)$ is the spectrum of $T$.
			\item the resolvent map $R: \rho(T) to L(X)$ is defined by $\lambda \mapsto (\lambda Id - T)^{-1}$
			\item the spectrum of $T$ is divided into \[\sigma(T) = \sigma_p(T)\cup \sigma_C(T) \cup \sigma_r(T)\] where \begin{itemize}
				\item $\sigma_P(T) = \{\lambda \in \sigma(T): \ker (\lambda Id - T) \neq \{0\}\}$ is the point spectrum
				\item $\sigma_C(T) = \{\lambda \in \sigma(T)\setminus \sigma_P(T): Im (\lambda Id- T) \neq X \text{ but } \overline{Im (\lambda Id - T)} = X\}$
				\item $\sigma_r(T) = \sigma(T) \setminus (\sigma_P(T) \cup \sigma_C(T))$.
			\end{itemize}
			\item the elements of the point spectrum are called eigenvalues and $x \in X\setminus \{0\}: (I\lambda Id - T)(x) = 0$ is called eigenvector associated ot $\lambda$.
		\end{itemize}
	\end{definition}
	\begin{theorem}
		For $T\in L(X)$ \begin{enumerate}
			\item $\rho(T)$ is open.
			\item $\sigma(T)$ is compact and \[\sup_{\lambda \in \sigma(T)} \left|\lambda\right| \leq \lim_{m \to \infty} ||T^m||^{\frac{1}{m}} = r(T)\] is the spectral radius. In particular $r(T) \leq ||T||$
		\end{enumerate}
	\end{theorem}
	\subsection{Spectral theorem for self-adjoint compact operators}
	Let $X$ be a Hilbert space.
	\begin{lemma}
		Let $T \in K(X)$ self-adjoint. Then $||T||$ or $-||T||$ is an eigenvalue of $T$. 
	\end{lemma}
	\begin{lemma}
		Let $T \in L(X)$ be self-adjoint. Then all eigenvalues are real and eigenvectors corresponding to different eigenvalues are orthogonal.
	\end{lemma}
	\begin{lemma}
		Let $T \in L(X)$. If $M \subset X$ is a closed subspace s.t. $TM\subset M$, then $M^\bot$ is invariant under $T^*$.
	\end{lemma}
	\begin{theorem}
		Let $X$ be a Hilbert space, $T\in K(X)$ self-adjoint. Then $\exists$ ONS $ (\phi_n)_{n \in I} \subset X, \; I\subset \mathbb{N}, \text{ and } \exists (\lambda_n)_{n \in I} \subset \mathbb{R}$ s.t. $\forall x \in X$ \[Tx = \sum_{n \in I} \lambda_n \langle x, \phi_n\rangle \phi_n\] In particular $T\phi_n = \lambda_n \phi_n \; \forall n \in \mathbb{N}$. If $I$ is infinite, then $\lambda_n \to 0$.
	\end{theorem}
	\begin{corollary}
		Let $X$ be a separable Hilbert space with $\dim X = \infty$ and $T \in K(X)$ self-adjoint. Then $\exists$ a complete ONS $(e_n)_{n \in \mathbb{N}}$ of eigenvectors of $T$. In particular $\forall x \in X$ \[Tx = \sum_{n=1}^\infty \lambda_n \langle x,e_N\rangle e_n\]
		with $\lambda_n$ being the corresponding eigenvalue to $e_n$.
	\end{corollary}
	\section{Reproducing kernel Hilbert spaces}
	\subsection{Definitions}
	Here, we again use $\mathbb{K} = \mathbb{R}$ or $\mathbb{C}$. Further $X\neq \varnothing$ is simply a set. Also \[F(X,\mathbb{K}) = \{f: X \to \mathbb{K} \text{ a map}\}\]
	This is a vector space.
	\begin{definition}
		$H\subset F(X,\mathbb{K})$ is a reproducing kernel Hilbert space (RKHS) on $X$ if \begin{enumerate}
			\item $H$ is a subspace of $F(X,\mathbb{K})$
			\item $\exists \langle \cdot , \cdot \rangle: H\times H \to \mathbb{K}$ inner product, s.t. $(H,\langle \cdot, \cdot \rangle)$ is a Hilbert space
			\item $\forall x \in X$ the linear map $E_x: H\to \mathbb{K}$ with $E_x(f) = f(x)$ (the evaluation operator) is well-defined and bounded. 
		\end{enumerate}
	\end{definition}
	Let $\Omega \subset \mathbb{R}^n$ open, $H = L^2(\Omega)$ is not a RKHS since evaluation at a point does not make sense for $f \in L^2(\Omega)$.\\
	If $H$ is a RKHS, the evaluation operator $E_x \in H' \; \forall x \in X$. For $x \in X$, by Riesz-Fréchet $\exists! k_x \in H$ s.t. $E_x(F) = \langle f,k_x\rangle \; \forall f \in H$.
	\begin{definition}
		The function \[K:X\times X \to \mathbb{K}\]\[(x,y) \mapsto \langle k_y, k_x\rangle\]
		is called reproducing kernel of $H$. 
	\end{definition}
	\begin{remark}
		For $x,y \in X$ and $\mathbb{K} = \mathbb{C}$ \[K(x,y) = \langle k_y,k_x\rangle = \overline{\langle k_x,k_y\rangle} = \overline{K(y,x)}\] while if $\mathbb{K} = \mathbb{R}$ the kernel is symmetric. Further \[||E_x||^2 = ||k_x||^2 = \langle k_x,k_x\rangle = K(x,x) \geq 0\]
	\end{remark}
	\subsection{Theorem of Moore-Aronszajn}
	\begin{lemma}
		Let $H$ be a RKHS on $X$ with kernel $K$. Then $\forall n \in \mathbb{N}$ and $\forall \{x_1,...,x_n\}\subset X$ the matrix \[(K(x_i,x_j))^n\] is a positive semidefinite matrix, i.e. \[\sum_{i,j=1}^n \alpha_j K(x_j,x_i)\overline{\alpha_i} \geq 0 \; \forall \alpha \in \mathbb{K}^n\]
	\end{lemma}
	\begin{theorem}[Moore-Aronszajn]
		Let $X \neq \varnothing$, $K:X\times X \to \mathbb{K}$ s.t. \begin{enumerate}
			\item if $\mathbb{K} = \mathbb{C} \; K(x,y) = \overline{K(y,x)}$ and if $\mathbb{K} = \mathbb{R} \; K(x,y) = K(y,x) $
			\item $K$ is positive semidefinite 
		\end{enumerate}
		Then there exists a (unique) RKHS on $K$ with kernel $K$. Notation: $H(K)$.
	\end{theorem}
	\subsection{An application}
	\underline{Interpolation:} Let $\{x_1,...,x_n\} \subset X$ be distinct points. $\lambda_1,...,\lambda_n \in \mathbb{C}$ not necessarily distinct. Let $H$ be a RKHS on $X$.\\
	\underline{AIM:} Find $f \in H$ s.t. the least square error \[J(f) = \sum_{i=1}^n \left|f(x_i) - \lambda_i\right|^2\] is minimal at $f$ and among all minimizers we want the one with minimal norm. 
	\begin{theorem}
		Let $H$ be a RKHS on $X$. $\{x_1,...,x_n\} \subset X$ distinct points in $X$. $A:= (K(x_i,x_j))$ a $n\times n$-matrix. $v = (\lambda_1,...,\lambda_n)^T \in \mathbb{K}^n$. Then $\exists w \in \mathbb{K}^n$ s.t. $v-Aw \in \ker(A)$ and \[H \ni f:= \sum_{i=1}^n w_ik_{x_i}\] satisfies \[J(f) = \inf_{g \in H} J(g)\] We have $k_{x_i} = K(\cdot,x_i)$ and $f$ is the unique minimizer of minimal norm. 
	\end{theorem}
	\section{Theorems on continuous linear maps}
	\subsection{uniform boundedness}
	We need the theorem of Baire a lot in this chapter, so we recall it.
	\begin{theorem}[Baire's theorem]
		Let $(X,d)$ be a complete metric space and $(U_n)_{n \in \mathbb{N}}$ s.t. $U_n \subset X$ is open and dense $\forall n \in \mathbb{N}$. Then \[\bigcap_{n \in \mathbb{N}} U_n\] is dense in $X$.
	\end{theorem}
	\begin{corollary}
		Let $(X,d)$ be a complete metric space, $(A_n)_{n \in \mathbb{N}}$ s.t. $A_n$ closed $\forall n \in \mathbb{N}$ and $X = \bigcup_{n \in \mathbb{N}}A_n$. Then $\exists N \in \mathbb{N}$ s.t. $A_N$ has an interior point.
	\end{corollary}
	\begin{theorem}[uniform boundedness principle]
		Let $X \neq \varnothing$ be a complete metric space, $Y$ a normed space. Let $F \subset C^0(X,Y)$ s.t. $$\sup_{f \in F} ||f(x)||_Y < \infty \; \forall x \in X$$ Then $\exists x_0 \in X$ and $\exists r_0 > 0$ s.t. \[\sup_{x \in \overline{B_{r_0}(x_0)}} \sup_{f \in F} ||f(x)||_Y < \infty\]
	\end{theorem}
	\begin{theorem}[Banach-Steinhaus]
		Let $X$ Banach space, $Y$ normed space, $\mathcal{T} \subset L(X,Y)$ family such that \[\sup_{T \in \mathcal{T}} ||Tx||_Y < \infty \; \forall x \in X\] Then $\mathcal{T}$ is a bounded set in $L(X,Y)$ i.e. \[\sup_{T \in \mathcal{T}} ||T||_{L(X,Y)} < \infty\]
	\end{theorem}
	\begin{lemma}
		Let $X$ be a Banach space, $Y$ a normed space, $(T_n)_{n \in \mathbb{N}} \subset L(X,Y)$ s.t. $\forall x \in X$, $T_nx$ converges in $Y$. Then $T:X\to Y$ with $x \mapsto \lim_{n \to \infty} T_nx$ is linear and continuous.
	\end{lemma}
	\subsection{open map theorem}
	\begin{definition}
		Let $(X,d_X)$, $(Y,d_Y)$ be open metric spaces and $f: X \to Y$. Then $f$ is called open if $\forall U \in X$ open $f(U) \subset Y$ is open.
	\end{definition}
	\begin{remark}
		Let $f: X \to Y$ be bijective. Then $f$ is an open map iff $f^{-1}$ is continuous.
	\end{remark}
	\noindent\underline{Attention:} $f$ continuous and bijective $\not \implies$ $f^{-1}$ is continuous. A counterexample is $f: [0,1] \cup (2,3] \to [0,2]$ where \[f(x) = \begin{cases}
		x, & x \in [0,1]\\
		x-1, & x \in (2,3]
	\end{cases}\]
	\begin{lemma}
		Let $T: X \to Y$ be linear, $X,Y$ normed spaces. \begin{enumerate}
			\item $T$ is open iff $\exists \delta > 0$ s.t. $T(B_1(0)) \supset B_\delta(0)$
			\item $T$ open $\Rightarrow$ $T$ is surjective
		\end{enumerate}
	\end{lemma}
	\begin{theorem}[open map theorem]
		If $X$, $Y$ are Banach spaces, $T \in L(X,Y)$ s.t. $T$ surjective, then $T$ is open.
	\end{theorem}
	\begin{theorem}[theorem of the inverse]
		Let $X$, $Y$ be Banach-spaces, $T \in L(X,Y)$ bijective, then $T^{-1}$ is continuous and in fact $T^{-1} \in L(Y,X)$.
	\end{theorem}
	\begin{corollary}
		Let $X$, $Y$ be Banach. Then any bijective map $T\in L(X,Y)$ is an isomorphism.
	\end{corollary}
	\begin{remark}
		$T\in L(X)$ where $X$ Banach then \[\rho(T) = \{\lambda \in \mathbb{K}: (\lambda ID - T)^{-1} \in L(X)\} = \{\lambda \in \mathbb{K}: \lambda Id - T \text{ bijective}\}\]
	\end{remark}
	\begin{theorem}
		Let $X,Y$ be Banach. Then $\mathcal{S} = \{T \in L(X,Y): T \text{ surjective}\}$ is open in $L(X,Y)$.
	\end{theorem}
	\subsection{Closed graph theorem}
	We work with the graph of an operator. Recall that, given $(X, ||\cdot ||_X), (Y, ||\cdot ||_Y)$, we can look at the normed space $X\times Y$ equipped with $||(x,y)||_{X\times Y} = ||x||_X + ||y||_Y$.
	\begin{definition}
		Let $T:X \to Y$ linear. \begin{enumerate}
			\item $G(T) = \{(x,y) \in X \times Y: y = Tx\}$ is the graph of $T$
			\item $T$ is called a closed linear operator if $G(T)$ is closed.
		\end{enumerate}
	\end{definition}
	\begin{remark}
		\begin{itemize}
			\item If $X,Y)$ are Banach spaces, then so is $X\times Y$
			\item $G(T)$ is a subspace of $X\times Y$ and in particular a Banach space
		\end{itemize}
	\end{remark}
	\begin{lemma}
		$T$ is a closed linear operator $\iff \; \forall (x_n)_{n \in \mathbb{N}} \subset X$ s.t. $x_n \to x$ and $Tx_n \to y$, then necessarily $Tx = y$. 
	\end{lemma}
	\begin{theorem}[closed graph theorem]
		Let $X$ and $Y$ Banach, $T:X\to Y$ linear. Then $T$ is a linear closed operator iff $T$ is continuous (bounded).
	\end{theorem}
	\begin{remark}
		If $X,Y$ Banach, $T:X \to Y$ linear, then $T$ is continuous \begin{itemize}
			\item iff $\forall (x_n)_{n\in \mathbb{N}} \subset X$ s.t. $x_n \to x$ in $X$ then $Tx_n \to Tx$ in $Y$
			\item iff $\forall (x_n)_{n \in \mathbb{N}} \subset X$ s.t. $x_n \to x$ and $Tx_n \to y$.
		\end{itemize} 
	\end{remark}
	\begin{definition}
		Let $X,Y$ be normed spaces and $D\subset X$ a subspace. $T:D\to Y$ linear is called closed if $\forall (x_n)_{n \in \mathbb{N}} \subset D$ s.t. $x_n \to x$ and $Tx_n \to y$ then $x \in D$ and $Tx = y$.
	\end{definition}
	\begin{lemma}
		Let $X,Y$ be Banach spaces, $D\subset X$, $T:D \to Y$ linear and clsoed. Define \[||\cdot ||_T : D \to [0,\infty)\] where \[||x||_T = ||x||_X + ||Tx||_Y\] called the graph norm. Then $||\cdot ||_T$ is a norm, $(D||\cdot||_T)$ is a Banach space and \[T: (D,||\cdot||_T) \to (Y,||\cdot||_Y)\] is continuous.
	\end{lemma}
	\subsection{Consequences}
	A central question in mathematics concerns the solvability of equations. Let $X,Y$ be any sets and $f:X\to Y$. Given $y \in Y$ is there an $x \in X$ s.t. $f(x) = y?$.\\
	Here $x$ and $Y$ are normed spaces and $T:X\to Y$ linear. The open map theorem implies that for Banach spaces $X$ and $Y$, $T:X\to Y$ linear bijective and continuous, then $T^{-1}:Y \to X$ is also continuous. As a consequence, the solution of $Tx = y$ depends continuously on $Y$. Further $\{T \in L(X,Y): T \text{ surjective}\}$ is open in $L(X,Y)$, when $X$ and $Y$ are Banach. With the Neumann series, we get
	\begin{theorem}
		If $X,Y$ Banach, \[A = \{T \in L(X,Y): T \text{ is an isomorphism}\}\] is open in $L(X,Y)$. I.e. if $T\in L(X,Y)$ isomorphism $\Rightarrow \exists \rho > 0$ s.t. $\forall S \in L(X,Y)$ s.t. $||S-T|| < \rho$ then $S$ is an isomorphism. 
	\end{theorem}
	\section{$L^p$-spaces}
	\subsection{Definitions}
	Let $(\Omega, \mathcal{A}, \mu)$ be a measure space.
	\begin{definition}
		$$\mathcal{L}^p(\Omega, \mu) = \{f \in \mathcal{M}(\Omega, \mathbb{R}): |f|^p \; \mu-\text{integrable}\}$$ for $1\leq p < \infty$ and \[\mathcal{L}^\infty = \{f \in \mathcal{M}(\Omega, \mathbb{R}): \exists N \in \mathcal{A}: \, \mu(N) = 0: \, \sup_{x \in \Omega\setminus N} |f(x)| < \infty\}\]
		We define the functions \[||f||_p = \left(\int_{\Omega}|f(x)|^p \, d\mu\right)^{1/p}\] and \[||f||_\infty = \text{esssup} |f| = \inf_{N \in \mathcal{A}, \mu(N) = 0} \left(\sup_{x \in \Omega\setminus N}|f(x)\right)\]
	\end{definition}
	\begin{lemma}
		For $p \in [1,\infty]$, $\mathcal{L}^p(\Omega, \mu)$ are vector spaces. The Hölder and Minkowski inequalities hold. But $||f||_p = 0 \not \Rightarrow f \equiv 0$. In general, only $f = 0$ $\mu$-a.e.
	\end{lemma}
	We define the equivalence relation $\sim$: $f\sim g$ iff $f = g$ $\mu$-a.e.
	\begin{definition}
		For $p \in [1,\infty]$ \[L^p(\Omega, \mu) = \mathcal{L}^p/\sim = \{[f]: f \in \mathcal{L}^p\}\]
	\end{definition}
	\begin{theorem}[Fischer-Riesz]
		For $p \in [1,\infty]$, $(L^p(\Omega, \mu), ||\cdot ||_p)$ is a Banach space. For $p = 2$, $L^2$ is a Hilbert space where \[\langle f,g\rangle = \int_{\Omega} f(x)g(x) \, d\mu(x)\]
	\end{theorem}
	\begin{remark}
		If $(f_k)_{k \in \mathbb{N}}$ Cauchy in $(L^p(\Omega, \mu), ||f||_p)$ then $\exists f \in L^p(\Omega, \mu)$ s.t. $f_k \to f$ in $L^p(\Omega, \mu)$ $\not \Rightarrow f_k \to f$ pointwise $\mu$-a.e.\\
		But $\exists$ subsequence $f_{k_j} \to f$ $\mu$-a.e.
	\end{remark}
	\subsection{Approximation in $L^p$}
	In $\mathbb{R}^n$ with Lebesgue measure: $\Omega \subset \mathbb{R}^n$ measurable, $L^p(\Omega) = L^p(\Omega, \lambda^n)$.
	\begin{definition}
		For $f:\Omega \to \mathbb{R}$ continuous \[\text{supp}(f) = \overline{\{x \in \Omega: \, f(x) \neq 0\}}\] is called the support of $f$.
	\end{definition}
	\begin{definition}
		Let $C^0_0(\Omega, \mathbb{R}) = \{f: \Omega \to \mathbb{R}: f \text{ is continuous and } supp(f) = k \text{ compact and } \exists \Omega' \subset \Omega \text{ open s.t. } k \subset \Omega'\}$ the space of continuous functions with support compactly contained in $\Omega$.
	\end{definition}
	\begin{theorem}
		Let $\Omega \subset \mathbb{R}^n$ open, $1\leq p < \infty$. Then $C^0_0(\Omega)$ is dense in $L^p(\Omega)$
	\end{theorem}
	\begin{definition}
		Similarly we define \[C_0^k = \{f: \Omega \to \mathbb{R}: \, f \in C^k(\Omega) \text{ and } f \in C_0^0(\Omega; \mathbb{R})\}\] the space of $k$-times continuously differentiable functions with compact support in $\Omega$ and $C_0^\infty(\Omega) = \bigcap_{k \in \mathbb{N}} C_0^k(\Omega)$ called the set of test functions.
	\end{definition}
	\begin{definition}
		Define $\phi: \mathbb{R}^n \to \mathbb{R}$ where \[\phi(x) = \begin{cases}
			c\cdot \exp(-\frac{1}{1-||x||^2}), & ||x||<1\\
			0, & \text{ otherwise}
		\end{cases}
		\]
		Where $c>0$ is s.t. \[\int_{\mathbb{R}^n} \phi(x)\,dx = 1\]
		Further, for $\varepsilon>0$, $\phi_\varepsilon:\mathbb{R}^n \to \mathbb{R}$ is defined by \[\phi_\varepsilon(x) = \frac{1}{\varepsilon^n} \phi\left(\frac{x}{\varepsilon}\right)\]
		Then $\int_{\mathbb{R}^n} \phi_\varepsilon(x)\,dx = 1$.
	\end{definition}
	\begin{definition}
		For $f\in L^1(\Omega), \, \varepsilon>0$ and $f_\varepsilon: \mathbb{R}^n \to \mathbb{R}$ be defined by \[f_\varepsilon(x) = \int_{\Omega} \phi_\varepsilon(x-y)f(y)\,dy\] called the smoothing of $f$.
	\end{definition}
	\begin{remark}
		This is an example of a convolution. For $f,g:\mathbb{R}^n \to \mathbb{R}$ integrable \[\int_{\mathbb{R}^n} f(x-y)g(y)\,dy = f*g(x) = g*f(x)\] is the convolution of $f$ and $g$
	\end{remark}
	\begin{lemma}
		Let $\Omega \subset \mathbb{R}^n$ open $f \in L^1(\Omega)$, $\varepsilon>0$. Then $f\varepsilon \in C^\infty(\mathbb{R}^n)$. If $supp(f) = k\subset\Omega$ compact then for $\varepsilon < dist(k,\partial \Omega)$, $f_\varepsilon \in C_0^\infty(\Omega)$.
	\end{lemma}
	\begin{theorem}
		Let $\Omega \subset \mathbb{R}^n$ be open. \begin{enumerate}
			\item If $f \in C^0(\Omega)$, $K\subset \Omega$ compact, $f_\varepsilon \to f$ uniformly on $K$.
			\item If $f \in L^p(\Omega)$, $p\in [1,\infty)$, then $||f_\varepsilon||_p \leq ||f||_p$ and $f_\varepsilon \to f$ in $L^p(\Omega)$.
		\end{enumerate}
	\end{theorem}
	\begin{corollary}
		Let $\Omega \subset \mathbb{R}^n$ be open. Then $C^\infty_0(\Omega)$ is dense in $L^p(\Omega)$, $1\leq p < \infty$. I.e. \[\overline{C_0^\infty(\Omega)}^{||\cdot||_p} = L^p(\Omega)\] 
	\end{corollary}
	\begin{remark}[Dirac Sequences]
		$(\phi_k)_{k\in \mathbb{N}} \subset C^\infty(\mathbb{R}^n)$ s.t. \begin{itemize}
			\item $\int \phi_k dx = 1$
			\item $\forall \varepsilon>0 \; \lim\limits_{k \to \infty} \int_{\mathbb{R}^n\setminus B_\varepsilon(0)} \phi_k dx = 0$ 
		\end{itemize}
		allow for a generalization of the above theorem.
	\end{remark}
	\begin{definition}
		$L^p_{loc}(\Omega) = \{f \in L^0(\Omega) : f \in L^p(K)$ for all compact sets $K\subset \Omega\}$. And further $L^0(\Omega)$ is the space of equivalence classes of a.e. equal measurable functions from $\Omega \to \mathbb{R}$.
	\end{definition}
	\begin{theorem}[Fundamental Lemma in the calculus of variations]
		Let $\Omega \subset \mathbb{R}^n$ open, $f\in L_{loc}^1(\Omega)$ and \[\int_\Omega f(x)\phi(x)dx = 0 \; \forall \phi \in C_0^\infty(\Omega)\] then $f \equiv 0$ a.e.
	\end{theorem}
	\subsection{Separability of $L^p$}
	\begin{theorem}
		Let $\Omega \subset \mathbb{R}^n$ be open, $p \in [1,\infty)$. Then $L^p(\Omega)$ is separable.
	\end{theorem}
	\subsection{Dualspace of $L^p(\Omega)$, $p \in [1,\infty)$}
	Similar to $l^p$. Let $q \in (1,\infty]$ s.t. $\frac{1}{p}+\frac{1}{q} = 1$, $\Omega \subset \mathbb{R}^n$ open. Let $g \in L^q(\Omega)$. \[T_g:L^p(\Omega) \to \mathbb{R}, \; T_g(f) = \int_\Omega f(x)g(x)dx\]
	Then, by Hölder, \[T_g \in L^p(\Omega)', \,||T_g||_{L^p(\Omega)} \leq ||g||_q\]
	\begin{theorem}
		Let $\Omega \subset \mathbb{R}^n$ open, $p \in [1,\infty)$ and $q \in (1,\infty]$ s.t. $\frac{1}{p}+\frac{1}{q} = 1$. Then $J:L^q(\Omega) \to L^p(\Omega)$ with $g\mapsto T_g$ is an isometric isomorphism.
	\end{theorem}
	\begin{theorem}
		Let $(\Omega,\mathcal{A},\mu)$ be a $\sigma$-finite measure space and $\nu:\mathcal{A} \to \mathbb{R}$ a bounded signed measure, i.e. \begin{itemize}
			\item $\nu(\varnothing) = 0$
			\item $\nu$ is $\sigma$-additive
			\item the total variation $$||\nu||_{var} = \sup\{\sum_{k=1}^n \left|\nu(E_i)\right|: \, n \in \mathbb{N} \text{ and } E_1,...,E_n \in \mathcal{A} \text{ are pairwise disjoint sets}\}$$ is finite
		\end{itemize}
		Then the following are equivalent: \begin{enumerate}
			\item $\exists! f \in L^1(\Omega, \mu)$ s.t. $\nu(A) = \int_Af\,d\nu$
			\item $\nu$ is absolutely continuous w.r.t. $\mu$, i.e. \[\forall A \in \mathcal{A}: \mu(A) = 0 \Rightarrow \nu(A) = 0\]
		\end{enumerate}
	\end{theorem}
	\begin{remark}
		In 1, one often uses the notation $f = \frac{d\nu}{d\mu}$ and calls this function Radon-Nikodym derivative of $\nu$ w.r.t. $\mu$. 
	\end{remark}
\section{Reflexive Spaces and Weak Convergence}
	\subsection{Reflexive Spaces}
	Let $X \neq \{0\}$ be a normed space and $X'$ be its dual.
	\begin{definition}
		$X'' = (X')' = L(X',\mathbb{K})$ is the bi-dualspace of $X$.
	\end{definition}
	There is a natural map between $X$ and $X''$. This is $i_X: X \to X''$, defined by \[x\mapsto i_X(x) \in X'', \text{ i.e. } i_X(x):X'\to\mathbb{K}\]
	That is $i_X(x)(f) = f(x)\, \forall f \in X'$.\\
	$i_X$ is linear and bounded.
	\begin{definition}
		$i_X:X\to X''$ as above is called canonical evaluation map. 
	\end{definition}
\end{document}